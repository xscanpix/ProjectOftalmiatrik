 \documentclass[12pt,a4paper,notitlepage]{report}
\usepackage[utf8]{inputenc}
\usepackage{cite}
\usepackage{etoolbox}
\usepackage{listings}
\usepackage{graphicx}
\usepackage{subcaption}
\usepackage{amsmath}
\usepackage{titlesec}
\usepackage[nottoc, notlof]{tocbibind}
% might need later, for diagrams
%\usepackage{tikz}
%\usetikzlibrary{arrows,shadows}
%\usepackage{pgf-umlsd}
\usepackage{pgfplots}
\usepackage{siunitx}
\usepackage{multirow}

\def\checkmark{\tikz\fill[scale=0.4](0,.35) -- (.25,0) -- (1,.7) -- (.25,.15) -- cycle;}

%%%%%%%% Bibliografi %%%%%%%%%%%%%%%%%%%%%%%%%%%%%%%%%
%% Ändrar namnet på referens headern
\renewcommand\bibname{References}
%%%%%%%%%%%%%%%%%%%%%%%%%%%%%%%%%%%%%%%%%%%%%%%%%%%%%%

\titleformat{\chapter}[display]
  {\normalfont\bfseries}{}{0pt}{\Large}

%%%%%%%%%%%%%%%%% Start document %%%%%%%%%%%%%%%%%%%%%
\begin{document}
\pagenumbering{roman}

\title{Report Draft\\Digitalization of Visual Acuity Testing}
\author{Anders Ohlsson\\Christian Johansson\\Svante Nilsson}
\maketitle

\begin{abstract}
Digitalization of Visual Acuity Testing is a project involving development of a method for displaying optician's eye charts. The charts are going to be displayed on a digital screen and controled via an Android tablet. This solves several problems relating to lighting and distance from the patient to the chart, while also providing more ways for the optician to tailor the examination of the patient. The Digitalization of Visual Acuity Testing project will be improving upon a proof-of-concept system that has been used to test whether or not digital screens are a viable alternative to physical charts.
\end{abstract}

\tableofcontents
\listoffigures
\clearpage

%%%%%%%%%%%%%%%%%%%%%%%%  DEL 1: FÖRUTSÄTTNINGAR
\pagenumbering{arabic}
\setcounter{page}{1}
\chapter{Introduction}
\section{Background}
To measure a patients visual acuity opticians today use charts with printed lines of continuously smaller symbols. The measurement is made by having the patient read symbols on the chart from a set distance. The person reads down the chart until it is no longer possible to make out the characters. Based on the size of the character the patient failed to recognize, a visual acuity can be determined. 

For such charts to function as a basis for these standardized tests of visual acuity the testing environment must be carefully prepared. Among a few things, for the measurement to be optimal the chart must be placed at a distance of six meters from the patient, and the luminosity in the room must be at a specific level. In the event that the room is too small or the lighting insufficient, this can lead to irregularities or faulty measurements in some tests. 

These conditions cannot always be met. Sometimes the examination room is not sufficiently large for measurements from six meters. In the event that the room is too small or the lighting insufficient, this can lead to irregularities or faulty measurements in some tests.

There are also several different variations on these charts. Different sets of symbols, ranging from different subsets of latin letters to more abstract symbols, have been developed for different types of patients, such as those who are analphabetic, or for small children. The layout of these charts also have variations, from the classic Snellen Chart that was developed in the 19th century, to the more modern LogMAR chart.

\section{Related work}
Several pieces of software exists that tries to solve the problems of displaying information from an Android tablet or phone on a larger screen. 

\subsubsection{Chromecast and Apple TV}
The Chromecast \cite{chromecast} and Apple TV \cite{appletv} are small devices which are plugged into a TV and enables mobile phones and tablets to stream media to them. The devices are small and easy to handle and a lot of applications support streaming to them. The Chromecast is specifically made for Android devices and Apple TV for iOS device, but they both support each other.

\subsubsection{Digital Eye Chart}
 Digital Eye Chart \cite{digitaleyechart} is a visual acuity testing program that attempts to solve the problem of having several charts by using a monitor instead. The user can then, with the help of a computer, change the symbols on the chart and even generate a random chart. 

\subsubsection{iChart+}
iChartPlus \cite{ichartplus} is another program which is similar to Digital Eye Chart but also have an adjustable viewing distance. 

\subsubsection{AxAnIvIs}
AxAnIvIs is a prototype that is currently under testing at Uppsala University Hospital. The project attempts to improve on how we perform visual acuity tests by digitalizing them. It adds more types of charts than the standard physical ones that exists today. The system uses a webbrowser running on an Android tablet that is connected to a screen. A library of vector graphic files are displayed in the webbrowser. These files contains the different charts as well as some other specialized tools. Currently the Android tablet has to be held in a fixed position in the examination room, restrained by the cable going to the screen. The charts used are fixed and not very easily changed, which makes it very hard to try out new techniques and methods of measuring.

\subsubsection{Our project}
Most of the before mentioned systems have some form of negative side that we try to improve on, mostly problems about high cost and difficulties with extending the software. We are different with that we want to create a system that is cheap, easy to understand and easy to maintain. Creating a system that have all the previously mentioned sides is one of the goals we have. Another important goal is creating a system that really is speacialized on measurements of visual acuity. All these points are important to give the users the best experience.

\section{Purpose}
Having an efficient and accurate system for testing is important in any kind of environment. Especially in one that handles visual testing where an unaccurate test can result in possible injuries of some sorts. Therefore it is important to have a well-built background and accurate information. With the help of Professor %Här kan vi skriva om Per. 
X as our expert in the field we will try to create this system and on the way there talk about different problems that can arise. 

To solve the problems with this system the Department of Neuroscience have performed tests using a digital screen to present the charts instead of using a printed medium. This means the optician can have many more variations on the charts at disposal. %The set of symbols used on the charts can be switched within a menu, and the screen provides an more than adequately lit surface by itself. The current system is implemented by connecting an Android tablet to a screen via an HDMI cable, and the charts are rendered via a web browser on the tablet. Testing has proceeded for over a year, and has yielded very good results.

The goal of this project is to take the system used for these tests and develop an implementation of said system with several advantages over the testing system, such as:

\begin{itemize}
	\item An application for the tablet with an improved user interface, both aesthetically and functionally.
	\item Controlling the server over WiFi using the tablet, meaning the optician can move around the room more easily.
	\item Altering the scale of the rendered symbols so that the patient can be closer or further away from the.
	\item Being able to tailor the testing process to the patient by changing the sets of symbols displayed or removing clutter from the charts.
\end{itemize}
	
and other features to make the process easier for the optician and patient.

\section{Motivation}
Determining visual acuity is a very labor- and equipment-intensive process and it is not going to stop being so. More and more people are experiencing eye problems. \cite{vision_loss} Handling printed charts is slow and not as accurate as it can be. On the other hand, a digital screen has increased functionality, accuracy and efficiency. A digitalization also gives the user a greater possibility for extending the usage of the system, including adding new charts and optotypes. 

\section{Issues}
There are several technical problems that we need to solve. We need to design a small computer system that will store and present the desired images to the screen. Our current plan is using a Raspberry Pi, since it is a cheap and compact alternative to PCs that is more than powerful enough for our purposes. Though, since the Raspberry Pi in running on a ARM CPU there will be a few compatibility problems that we need to solve. For example is the Java graphics library JavaFX, that is going to replace the older Swing library, not very well supported by Raspberry Pi. Not being able to use the latest features can create problems in the future when some things get deprecated. Though if that happens it is probably not very hard to re-write the GUI to a supported JavaFX version.

Networking needs to be added to control the Raspberry Pi over WiFi from the tablet, and some features like waking the screen as the app is started can be considered. With networking we face a few problems. % How the server is controlled by the tablet? (Direct function calls, network polling, blabla)

Designing the Android app for the tablet will bring technical problems, both from the limitations of portable computers, and making our software adaptable enough to work on as many brands of tablets as possible. For example, the symbols we are to present are stored in the .svg (Scalable Vector Graphics) format, which is not natively supported on Android. Because of this we must find and work with a licensed third-party software library to render them. Other than that the Android tablet doesn't impose too many big problems. %% Fler problem som kan uppstå med Android?

Finally we should design a way for the database of symbols and chart layouts to be easily updated and synchronized between devices and some server storage. This will be the least important task for us since it we wan't to focus on having something to work with firstly. Updating and synchronizing can be done by hand locally.

\section{Ethical considerations} %vilka etiska frågeställningar finns (tänk t.ex. på användningen av resultatet)?
Our ethical obligations with this project is providing a functional and stable product. If our system does not perform to specification, a lot of patients could get erroneous measurements during testing, costing a lot of time and money to correct. This is very important to think about since we are dealing with treatment of people.

\chapter{Problem Description}
\section{Problem}
The problem with today's visual acuity tests are that they are very clumsy to work with. The charts are printed and contains only one type of symbol, or optotype. This could be letters for people with the ability to read or an 'E' shape turned in different ways which allows children who are unable to read, or illiterate people, to identify which symbol they see. These charts are also only designed to be viewed from one specific distance, which makes some rooms to small and unusable.

\section{Delimitations}
Due to limits in time and manpower, this project is not intended to be a complete working system to be sold, but rather a working prototype that can be used to evaluate digital visual acuity charts. In order to achieve this, as much as possible of Java's and Android's built in libraries will be used. As neither support svg files, external libraries will be used for this.

The project will neither evaluate a digital screens effect on the eye in contrast to a physical chart nor analyze why a ETDRS or Axanivis chart should have the layouts they have. As both charts are made after a very strict specification, no new chart layouts will be made. Not only could such charts be useless, but they could also be misleading and misused by future users that in the end, could result in patients getting an inaccurate result.

No security features, such as encrypted network traffic or protection against several users connecting to the same server at the same time, will be implemented. Instead, this is left to the user to make sure that a secure network is used.

No automatic update system will be included, but instead, the user have to manually update both the android application and the server.



%\subsection{en lista bara för att lista upp avgränsningar (ska bli flytande text)}
%att göra:
%\begin{itemize}
%\item
%\end{itemize}
%skriver om:
%\begin{itemize}
%\item inte en helt färdig produkt
%\item inte utveckla sätt att visa svg
%\item inte analysera hur ögat påverkas av en display istället för tavla
%\item inte implementera något om syn, ögat, etc förutom hur tavlorna är gjorda
%\item inte skapa egna tavlor
%\item ingen hänsyn till säkerhet (användare får se till att LANet är säkert)
%\item inga uppdateringar i detta projekt
%\end{itemize}
%möjliga saker att göra om tid finns
%\begin{itemize}
%\item Att ingen annan kan ansluta till servern som används atm
%\item Kod till servern
%\item grating
%\item det konstiga stjärngrejen i pers grejjer
%\end{itemize}



\section{Requirements}
For this project to be considered successful, the system must be able to display the visual acuity chart chosen by the user. This means that the user have to be able to control which chart type to display on the monitor from an Android device. The types of charts that should be included are the four Axanivis charts, the upper, the lower, the one line and the single optotype chart. The user also have to be able to choose if the current chart should use the Sloan letters, Tumbling E, HVOT or LEA optotypes.

It is very important that the chart on the monitor displays all optotypes with the correct sizes and positions. In order to do this, the user must to be able control the distance the patient is watching the monitor from. It is also very important that the correct sizes, in LogMAR, is displayed on the Android device.

The intended user for this system should not need any specific background knowledge about technology more than the basic use of Android device, i.e. be able to start the device, start an application and how to perform simple Android gestures etc. This does not mean that, though it's a goal, it has to be easy to install and set up. 

The Android application should be very user friendly and follow the Android design principles set up by Google \cite{android_design}. This makes the application behave similar to other Android applications which will result in users already familiar with Android, to get the expected result from their actions. 

The system should also allow for automatic server detection so that users don't have to manually connect by entering IP addresses. This allows a quick and easy way to connect to a server. In order for the server detection to be considered working properly, it should detect a server on a stable network atleast 95\% of the time.

As this project only aims for a working prototype of the system, it is very important that it is easy to continue the development and add more features without rewriting large parts of the code.

%As this project only aims for a working prototype, there are no requirements on how easy it will be to install or set up, but it requires that already set up system is very user friendly.

%
%\subsection{ska tas bort senare:}
%\begin{itemize}
%%\item prototyp som kan utvärdera både axanivis projekt och digitala syntavlor
%\item 
%\end{itemize}
%klart:
%\begin{itemize}
%\item systemet ska vara lätt att bygga vidare på
%\item kunna ställa in avstånd ifrån skärm
%\item visa chart på skärm
%\item visa rätt storlek, avstånd, etc på optotype enl. teori
%\item styra charten ifrån android över wifi (ETDRS, HVOT, etc...)
%\item automatiskt hitta en server
%\item användarvänligt
%\item android appens design ska följa androids designprinciper
%\item visa upp rätt logmar
%\end{itemize}

\section{Evaluation}
In order to evaluate if all requirements have been met, the system will undergo a few tests. The first tests goal is to make sure that the user actually can choose a specific chart with specific optotypes to be viewed from a specific distance. This is done by the developers going through all charts, with all types of optotypes and setting the distance for each one of them. To achieve this, table \ref{tab:test1_empty} is used to systematically walk through every setting.

The second test is to make sure that the optotypes have the correct sizes on the monitor. This test will done by, outside of the system, calculating the expected sizes of the optotypes in centimeters. The calculated sizes will then be compared to the sizes output by the system and see if they match. If this is a match, the optotypes will be measured by a physical measurement tool in order to double check the result.

The third test will determine how user friendly the system is. This will be done by letting a user with no previous knowledge of the system, attempt to connect to a server and perform some basic tasks like setting a specific chart with a specific optotype or setting a specific distance. If the user testing has no previous knowledge of visual acuity charts, a brief explanation will be given so that the test doesn't fail because the user, for example, doesn't know what "HVOT" is.

The last test is a quantitative test to see how well the server detection performs. The test involves the Android application trying to detect a server on a stable WiFi a 100 times and see how many times the server was detected. This test can be done several times and the average number of detection's in a 100 tries will determine its success rate. In order for the server detection to be considered to have met the requirements, atleast 90\% of the attempts should have found the server.

%\subsection{hur ska den utvärderas}
%\begin{itemize}
%\item storlek på optotyper ska beräknas och jämföras med systemets resultat
%\item optotyper mäts i verkligheten, med linjal el. liknandes
%\item skall hitta server X ggr av Y försök på stadigt nätverk
%
%\end{itemize}

\chapter{Theory}
\section{Visual Acuity}
The measurement of a persons visual accuity has an important constraint that needs to be followed. This constraint is the fact that our eyes works best physiologically from greater distances. On the other hand, at shorter distances our eyes tend to make what we see a bit unfocused. To make up for this shift in focus a minimum testing distance has to be enforced to ensure that the shift in focus stays as small as possible. This distance has been set to be six meters.

\subsection{Measurement distance}
The specific distance of six meters is derived from how our eyes bend light. From that distance or greater the rays of light entering our eyes are nearly parallel to eachother. It is important that the rays of light that hit the eyes are parallel because then the light focuses on a more focused area in the back of the eyes. This happens because of the convex shape of the eye's lens. A convex shaped lens focuses parallel light onto the focal point. Since a convex lens focuses parallel light to a focal point a concave lens instead spreads light coming from a focal point to parallel rays. Combining these two type of lenses we can create a more focused light. 

\subsection{Defocus correction}
To make up for the shift in focus a suitable lens is put in front of the eye at about the distance that you would wear glasses. The type of lens mostly depends on the shape of your eye, any eye disease you might have or the distance from the lens to the eye. An important property of the lenses is at what distance the focal point lays at. So, to create a more focused light, which is exactly what glasses do, we have to put a lens in front of the eye to create more parallel light. These lenses can either be plus or minus lenses, either spreading or contracting the light.

In optician's terms a plus glass is a lens that focuses light coming further away (addition) and a minus glass is a lens that focuses light from nearer (subtraction). A real world example of this is glasses helping to correct ~near- and farsightedness.

\subsection{Angular magnification}
% TODO: Explain using formulas how the angular magnification works
\begin{figure}[h]
\centering
\includegraphics[width=120mm]{images/Angular_magnification.png}
\caption{Angular magnification\label{angular}}
\end{figure}

Though using lenses is a way of corrected for testing at short distances, a few problems arise from this. Figure \ref{angular} shows the problem of angular magnification. 

The magnification is a problem is visual acuity testing because if the optotypes on the eye chart are enlarged then an error will show up in the results.

\newpage
\begin{figure}[h]
\centering
\begin{tabular}{| l | l | l | l |}
    \hline
    Distance to chart (m) & Addition (D) & Magnification (Rel.) & Error (logMar) \\ \hline
    4                     & 0.25         & 1.5                  & 0.17           \\ \hline
    2                     & 0.5          & 3.5                  & 0.48           \\ \hline
    1                     & 1            & 6.0                  & 0.78           \\ 
    \hline
    \end{tabular}
    \caption{Magnification due to addition for closer distance\label{magtable}}
\end{figure}

There is need to have a standard to follow on how much addition or subtraction needed for a specific distance. Figure \ref{magtable} shows what values to use when testing visual acuity at shorter distances than six meters. \cite{PGSoderberg}

\section{Chart Design}
The visual acuity charts this project are based on ETDRS (Early Treatment Diabetic Retinopathy Study) chart (fig. \ref{fig:etdrs_chart}) \cite{Ferris} and a project developed by PG Söderberg \cite{PGSoderbergOral} at Uppsala University Hospital. 

\begin{figure}[h]
\centering
\includegraphics[width=100mm]{images/etdrs_chart.jpg}
\caption{Example of a ETDRS chart \\ (Source: http://precision-vision.com/)\label{fig:etdrs_chart}}
\end{figure} 

\subsection{Optotypes}
An optotype is a symbol that is used by a visual acuity chart. The optotypes are all the same width and height and designed on five by five grid and the most common type of symbol to use are letters but can be different depending on the patient\cite{Colenbrander}.

\textit{Sloan letters} are ten optotypes that have been chosen because they are similarly difficult to read\cite{Ferris} and are used by most visual acuity charts\cite{Colenbrander}.

\textit{Landholt C} is a sloan \textit{C} that is rotated by 0, 90, 180 or 270 degrees and is usually used in a more scientific setting. The Landholt C is used as reference when deciding the how difficult it is to read a specific optotype\cite{Colenbrander}.

\textit{Tumbling E} is like the Landholt C, but with the letter \textit{E} instead of \textit{C}. This optotype can be used instead of the Sloan letters if the patient doesn't know how to read. The patient can instead show or tell the optician how the \textit{E} is rotated\cite{Colenbrander}.

\textit{HVOT} are the letters \textit{H},\textit{V},\textit{O} and \textit{T}. These letters are chosen because they look the same if they are mirrored on the horizontally.

\textit{LEA} are four symbols that look like a house, circle, box and apple. These symbols similar to HVOT since they look the same if you mirror them on the horizontal axis.

\begin{figure}[h]
\centering
\begin{subfigure}{.3\textwidth}
  \centering
  \includegraphics[width=.4\linewidth]{images/landholt_c_optotype.png}
  \caption{Landholt C}
  \label{fig:landholt_c}
\end{subfigure}%
\begin{subfigure}{.3\textwidth}
  \centering
  \includegraphics[width=.4\linewidth]{images/tumbling_e_optotype.png}
  \caption{Tumbling E}
  \label{fig:tumbling_e}
\end{subfigure}
\begin{subfigure}{.3\textwidth}
  \centering
  \includegraphics[width=.4\linewidth]{images/lea_optotype.png}
  \caption{LEA}
  \label{fig:lea}
\end{subfigure}
\caption{Some example optotypes}
\label{fig:optotypes_example}
\end{figure}

\subsection{ETDRS Chart}
The ETDRS chart consists of 14 rows with 5 optotypes on each row. The size of each optotype is measured in LogMAR (Logarithmic of the Minimum Angle of Resolution) which is the base 10 logarithm of the angle of one fifth of an optotype (eq. \ref{eq:logmar} and fig. \ref{fig:logmar_calculation}) \cite{Bailey}. The angle unit is measured in arcminutes, where one arcminute is $\dfrac{1}{60}$ degrees or $\dfrac{\pi}{10800}$ radians. If the angle v in figure \ref{fig:logmar_calculation} is equal to 8 arcminutes, then the size of the letter would be $log_{10}(8) \approx 0.903\ LogMAR$.

%\begin{figure}[h]
%\centering
%\includegraphics[width=100mm]{images/etdrs_chart.jpg}
%\caption{Example of a ETDRS chart \\ (Source: http://precision-vision.com/)%\label{fig:etdrs_chart}}
%\end{figure} 

\begin{equation}
	\begin{split}
  		LogMAR & = log_{10}(angle)
  	\end{split}
  	\label{eq:logmar}
\end{equation}

\begin{figure}[h]
\centering
\includegraphics[width=100mm]{images/logmar_calculation.png}
\caption{The angle v is used to calculate the size in LogMAR of the optotype.\label{fig:logmar_calculation}}
\end{figure} 

The optotypes on the top row of the chart is of the size 1.0 LogMAR and then each row below is decremented by 0.1 LogMAR to the last row which is -0.3 LogMAR. This means that each optotype is approximately 1.2589 the height of the optotypes on the row below (fig. \ref{fig:chart_size_plot}) \cite{Ferris}. The spacing between each optotype is equal to the width of an optotype from the same row and the spacing between each row is equal to the height of an optotype on the lower row (fig. \ref{fig:etdrs_chart_sizes}).

\begin{figure}[h]
\centering
\includegraphics[width=110mm]{images/etdrs_chart_sizes.png}
\caption{The sizes and positions of the first three optotypes of row $n$ and $n+1$ where $x_n \approx 1.2589 x_{n+1}$. \label{fig:etdrs_chart_sizes}}
%\caption{The gray squares represents optotypes that on the upper row has a width and height of $x_n$ and on the lower row $x_{n+1}$.  \label{fig:etdrs_chart_sizes}}
\end{figure} 

\begin{figure}[ht!]
\begin{tikzpicture}
    \begin{axis}[
      xmin=1,xmax=14,
      ymin=0,ymax=180,
      xlabel={row},
      xlabel near ticks,
      ylabel={pixels},
      ylabel near ticks
    ]
      \addplot[color=blue, mark=*, only marks] plot coordinates{(1,162)(2,128)(3,102)(4,81)(5,64)(6,51)(7,41)(8,32)(9,26)(10,20)(11,16)(12,13)(13,10)(14,8)};
    \end{axis}
\end{tikzpicture}
\caption{The height of optotypes on different rows when watching a 15.6 inch screen with a resolution of 1920x1080 pixels from 2 meters. \label{fig:chart_size_plot}}
\end{figure}


%\newpage
\subsection{Axanivis Charts}
The Axanivis charts are an extension to the ETDRS chart that adds two new chart layouts and digitalizes them so they can be displayed on a screen. Since the original ETDRS chart is made for a physical chart, it doesn't have an aspect ratio \cite{Ferris} that is close to a common monitor. To solve this, the original chart is split into an upper (fig. \ref{fig:etdrs_upper}) and a lower chart (fig. \ref{fig:etdrs_lower}). The upper chart display the 5 first rows and the lower chart display the remaining 9.

\begin{figure}[h]
\centering
\begin{subfigure}{.5\textwidth}
  \centering
  %\includegraphics[width=.5\linewidth]{images/etdrs_top.png}
  \includegraphics[width=50mm]{images/etdrs_top.png}
  \caption{Upper Chart}
  \label{fig:etdrs_upper}
\end{subfigure}%
\begin{subfigure}{.5\textwidth}
  \centering
  \includegraphics[width=50mm]{images/etdrs_bottom.png}
  \caption{Lower Chart}
  \label{fig:etdrs_lower}
\end{subfigure}
\caption{The upper and lower ETDRS charts}
\label{fig:etdrs_upper_lower}
\end{figure}

The two new chart layouts uses is constructed using the same principles, like height and spacing of optotypes, as the original ETDRS chart. But instead of having 14 rows of 5 optotypes each, the first new layout takes a single column of the ETDRS chart and rotates it 90 degrees (fig. \ref{fig:etdrs_one_line}) so it's all on one line. This makes it wider than it's high which suits a monitor better. This also allows the patient to read from left to right, instead of from top to bottom, which is the common way to read in Europe and America. Since the patient only have to read one letter of each size, this layout gives a quicker but more inaccurate result. This is useful to get a quick estimate on the visual acuity of the patient.

\begin{figure}[h]
\centering
\includegraphics[width=100mm]{images/etdrs_one_line.png}
\caption{The one line chart}
\label{fig:etdrs_one_line}
\end{figure}

Once an estimate visual acuity have been achieved from the one line chart, the user can select an optotype and display the second new layout, which only display one optotype at a time (fig. \ref{fig:etdrs_single}). This layout removes all clutter and allow the patient to easily focus on only a single optotype at a time. The user can then page between the different optotype sizes in order to determine the visual acuity.

\begin{figure}[h]
\centering
\includegraphics[width=90mm]{images/etdrs_single.png}
\caption{The single optotype chart}
\label{fig:etdrs_single}
\end{figure}




\chapter{Implementation}
\section{Methods}
\subsection{System structure}
The system is divided into three major parts, an Android application, a server and networking. The Android application and server is modeled after the MVC (Model-View-Controller) pattern, where the model is the logic and the view is the GUI (Graphical User Interface). The controller part is the part on the Android application that listens for button clicks and gestures. Usually the networking would fall under controller in the MVC pattern but was, in this system, chosen to be a separate larger structure. Figure \ref{system_structure} shows how the different parts are connected.

\begin{figure}[ht!]
\centering
\includegraphics[width=90mm]{images/system_structure.png}
\caption{Overview of the system structure\label{system_structure}}
\end{figure}

\subsection{Android application}
The Android application is written in Java, which is the official language for Android. The static parts of the GUI, which will never change unless application updates are released, are written in XML. This is similar to how HTML is written and is supported by Android natively.

The charts shown on the Android device are generated at runtime. If the charts would have been written in XML, there would be no guarantee that they would fit on the screen or keep the correct proportions on tablets with different dimensions. It also avoids hard to read code as the large chart, for example, would need 70 defined areas for images, all with a unique text string id. The chart layouts are all drawn at the same time and in the same place, but sets the active chart visible and the rest invisible. As Android does not natively have support for vector graphics and the chart optotype images are vector images the AndroidSVG library is used to display them \cite{AndroidSVG}.

The size of the optotypes is calculated using equation \ref{eq:optotype_size}, where $d$ is the distance in centimeters, $v$ is the angle in radians and $h$ is the height of the optotype in centimeters.


\begin{equation}
	\begin{split}
  		2d\ tan(\frac{v}{2}) & = h
  	\end{split}
  	\label{eq:optotype_size}
\end{equation}

The settings associated to each chart are also generated at runtime. If this were not the case, every different version of the chart would need its own XML file, as each version contains a different number of predefined charts.

In order to control the application, \textit{controllers} where created in order to listen to button clicks and gestures when the user attempts to do something.

The design of the application follows the Android design principles by using Androids built in navigation drawer and the built in views i.e. buttons, radio buttons, lists etc. \cite{android_design}.

\subsection{Server application}
The server application is written in Java and uses the built-in \textit{Swing} library for drawing the GUI and the images. Swing is a very flexible and easy-to-use library with a lot of great tools for creating a good looking GUI. In our case
we won't be needing a very advanced GUI since the main purpose of the server is to render a fullscreen image at all times. This greatly simplifies the implementation of the GUI. Other than the rendering of images, the GUI will consist of a small network controller where the user can specify a number of settings. These settings will be needed to configure for example what name the server will have on the network. It's also neccessary to have settings for the location of the server.

The rendering of images, specifically SVG vector graphic images, will be handled by an \textit{~Apache} library called \textit{Batik}. It is well supported and extensive library for rendering a multitude of different file formats. The SVG files contain information about the optotypes used on the charts. Using procedural positioning of the SVG images, we can easily create a well structured image to render to the screen.

The server will also handle the networking on its side, connecting to the Android tablet over a TCP connection. This will be done using a third-party library \textit{Kryonet}. \cite{kryonet} Kryonet is a library that wraps Java's own networking functionality and improves upon it. The library's strong-point is the way it implements serialization of Java objects. Using Kryonet we can without a lot of effort serialize objects, send the object over the network and then deserialize is, getting back the original object with all of its internal structure. 

Network communication will be handled using a submit-poll system. The networking part of the server will recieve information from the network and submit it to a thread-safe queue. While this happends, the GUI polls the information from the queue and handles the information. This methods is neccessary because of how Swing is constructed. Swing runs on a dispatch-thread basis, meaning only the dispatch-thread itself can change things in the GUI. Trying to change things in the GUI from another thread would cause a lot of problems. Now the GUI can try to poll whenever it has time, though try to do this on a period-base.  

\subsection{Network}
The network communication between the client and the server consists of two parts. The first part is server detection and the second is server communication. All network communications are handled in threads separate from the main thread as the network operations would halt the flow of the program otherwise. Once the operations are complete and the main threads finds it convenient, it will grab the result from the network threads.

\subsubsection{Server detection}
In order to detect all servers on the network, the Android application sends out a discovery request message 5 times with a 1 second interval using UDP (User Datagram Protocol) on the networks broadcast address. The servers, which constantly listens for broadcasts containing the correct message, will then respond to that message with it's name, again using UDP. Once the Android application receives a response it adds the server to a list. The user can then choose to connect to one of the found servers or attempt the detection again if no servers where found.

Since UDP is a connectionless protocol, there is no guarantee that the server actually receives the message, the discovery request message is sent 5 times, which means the server have 5 chances to receive the message and respond to it. This means that the server will receive a discovery request 0 to 5 times and respond to it the same number of times. To avoid duplicate server entries in the list, a server is only added if it doesn't already exist in the list.

%In order to detect all servers on the network, the Android application sends out a UDP (User Datagram Protocol) broadcast message containing server discovery request. The application will then listen for incoming server responses. Once the application haven't received anything for three seconds from the last server response, it will stop listening and determine that all available servers have responded. The names and addresses of the servers will then be extracted from each response message and stored in a list so the user can choose which one to connect to.

%The servers themselves are constantly listening for broadcasts containing the server discovery request. Once a request is received, the server will send a response and then keep listening for more requests.

\subsubsection{Server communication}
The communication between the Android application and the connected server is handled using strings of characters divided to three lines using line separators. The first line consists of a check word, which is the same for every message. If this check word is wrong or doesn't exist, the server will ignore the message. The second line is the command to the server. This could be a command to set the currents chart type or size to something else. The third line is the data required for the command. The data can be empty (null) or a string of characters. Depending on the command, the string of characters can be read as either a sequence of ascii characters or a sequence of numbers.

The messages are sent using a TCP (Transmission Control Protocol) connection to the server, which guarantees that the entire message will be correct and that it will be retrieved by the server as long as the server exists on the network. If there are no servers available on the network, the user will be informed and have to reconnect to a server.

%The communication between the client and server is handled using simple TCP (Transimission Control Protocol) messages. The TCP messages are used for sending and receiving commands and are structured as three lines of text (Example \ref{tcp_message}). The first line is a \textit{check} word and is the same for every \textit{command}. The second line is a \textit{command} word which tells the receiver what to do and the third line is the \textit{data} required for the \textit{command}. In many cases the simple \textit{command} is enough and the \textit{data} is left empty.

%\begin{center}
%\renewcommand{\lstlistingname}{Example}
%  \lstset{%
%    title=Example of TCP message,
%    basicstyle=\ttfamily\footnotesize\bfseries,
%    xleftmargin=.2\textwidth, xrightmargin=.2\textwidth
%  }
%\begin{lstlisting}[caption=TCP Message, label=tcp_message]
%CHECK_WORD
%LARGE_CHART
%NULL
%\end{lstlisting}
%\end{center}

\begin{figure}[h]
  \lstset{%
    basicstyle=\ttfamily\bfseries,
    xleftmargin=.4\textwidth, xrightmargin=.2\textwidth
  }
\begin{lstlisting}
AXANIVIS
DISTANCE
500
\end{lstlisting}
\caption{A message with the check word \textit{AXANIVIS}, the command \mbox{\textit{DISTANCE}} and the data \textit{500}. This message would make the server to resize the currently active chart to be suitable for a user watching from 500 centimeters. \label{fig:example_message}}
\end{figure}

%\label{system_structure}



 %where the server is connected to a large screen and is controlled by the Android application over a local network. The server and


%The communication between the Android tablet and the server will be handled using a third-party network library called Kryonet \cite{Kryonet}. 

%\section{System structure} %%%%%%%%%%%%%%%%%%%%%%%%Systemstruktur: vilka delar kan ni se att ert system/problem består av? (T.ex. databas, webbinterface, AI-modul, grafik...)
%The system is split up between an Android application and a PC application. These applications will both be structured with the Model-View-Controller design pattern as a base in order to split up user input, graphics and logic. The PC application will act as a server and display an eye chart on a larger screen connected with an HDMI-cable. The Android applications purpose is to control the PC-application over a WiFi network and make sure the correct chart is displayed on the larger screen. The Android application will display the same image shown on the large screen, but with some other extra information that's useful for the optician.

%\textbf{Image of System structure TBI}

%The dicrete parts of this project are:

%\begin{itemize}
%	\item Android application
%	\begin{itemize}
%		\item User Interface
%		\item Backend/Network
%	\end{itemize}
%	\item Screen controller
%	\begin{itemize}
%		\item Display
%		\item Backend/Networking
%	\end{itemize}
%\end{itemize}





%\section{Boundaries (OLD!)} 
%%%%%%%%%%%%%%%%%%%%%%%%%Avgränsningar: vad ska INTE göras, även om man kanske kunde tro det?
%We are not tasked to develop a library to display .svg files, and will be using a licensed one. We are also not tasked to populate the database with symbols and layouts, but instead provide a comprehensive API so that clients can create their own.
%
%%%%%%%%%%%%%%vad ska bara göras om tid/resurser/omständigheter räcker till?
%
%There are however several extensions to the system that can be considered if we have time to spare. We can extend the system to support more types of charts, such as charts for testing colour blindness and stereoscopic vision, which also rely heavily on rooms with appropriate lighting.
%
%While it deviates further from the main part of the project, we were asked by our employer to consider implementing the ability to control devices such as lights around the room from inside the app. This could probably be done, and we will look into it if time allows.
%
%\section{Demands (OLD!)} %%%%%%%%%%%%%%%%%%%%%%%%Krav: vilka krav ställer ni/andra på ert resultat? hur snabbt? hur många användare? hur strömsnålt? eller vad som är relevant
%There are a few core demands that our product needs to meet. First and foremost, it needs to work, and do what the current testing system can already do. Second, it needs to be easy to use and have an intuitive interface. Third, it should be very energy efficient, since forcing the user to recharge the tablet often reduces productivity. 
%
%\section{Evaluation (OLD!)} %%%%%%%%%%%%%%%%%%%%%%%%Utvärdering: hur ska ni utvärdera ert arbete/system, hur vet ni om/hur bra ni lyckats?
%The easiest way to see if our product satisfies these demands is putting it into the hands of the opticians at the university hospital and letting them test it, recording bug reports or comments. Testing will also let us determine if the system is as energy efficient as we intended.

\chapter{Results}
\section{User Interface}
\subsection{Android application}
The finished Android application turned out as intended and uses Androids standard GUI elements. To navigate between the connection settings and different type of charts, Androids built in navigation drawer is used (fig. \ref{fig:app_nav_drawer}). In order to reach the connection settings, the user must click the name of the connected server, in the case that no server is connected, it will, instead of the name of a server, say \textit{Click to connect}.

\begin{figure}[ht!]
\centering
\includegraphics[width=56mm]{images/appgui/nav_drawer.png}
\caption{The Android application navigation drawer used for navigating the application. a) Name of the application b) The currently connected server c) Available optotypes d) Grating, not yet implemented e) Test, used for testing new functions}
\label{fig:app_nav_drawer}
\end{figure}

Once the connection settings page have been reached, the user will be shown the connection status and a \textit{scan network} button (fig. \ref{fig:scan1}) which can be used to scan the network for servers. Once the button is pressed, the button will be disabled and found servers will appear in a list below (fig. \ref{fig:scan2}). When the desired server have been found, the user can click on it to connect to it and once the entire scan is complete, the \textit{scan network} button will be enabled again.

\begin{figure}[ht!]
\centering
\begin{subfigure}{.5\textwidth}
  \centering
  %\includegraphics[width=.5\linewidth]{images/etdrs_top.png}
  \includegraphics[width=50mm]{images/appgui/scan1.png}
  \caption{No connection}
  \label{fig:scan1}
\end{subfigure}%
\begin{subfigure}{.5\textwidth}
  \centering
  \includegraphics[width=50mm]{images/appgui/scan2.png}
  \caption{Scan button pressed}
  \label{fig:scan2}
\end{subfigure}
\begin{subfigure}{.5\textwidth}
  \centering
  \includegraphics[width=50mm]{images/appgui/scan3.png}
  \caption{Connected to server}
  \label{fig:scan3}
\end{subfigure}
\caption{Server connection steps}
\label{fig:scan}
\end{figure}

Once the application has connected to a server, the user can navigate to the chart that should be viewed on the screen. This is done by opening the navigation drawer, either by swiping from the left edge of the screen to the right, or by pressing the navigation drawer icon in the top left corner. When the navigation drawer is open, the user can choose between the \textit{ETDRS} (Sloan letters), \textit{Tumbling E}, \textit{HVOT} or \textit{Icons} (LEA) to display that chart (fig. \ref{fig:etdrs_large}). 

\begin{figure}[ht!]
\centering
\includegraphics[width=120mm]{images/appgui/etdrs_large.png}
\caption{The upper large ETDRS 1 chart displayed with each rows LogMAR values to the far left.}
\label{fig:etdrs_large}
\end{figure}

The user can now choose which chart version, size and distance to use. The chart version decides which optotype to use at which position. The ETDRS chart have 3 premade versions\cite{Ferris} and the rest of the charts have 1. There is also an option to use a randomly generated version which is generated when the application start. If this option is selected, the generate button below enables, so that the user can generate a new version if necessary. The random version is generated with no restrictions and should be used with caution.

The user can also choose which chart size, \textit{Large}, \textit{Medium} or \textit{Small}, to use. The large chart is the same chart as the \textit{upper} and \textit{lower ETDRS} charts (fig. \ref{fig:etdrs_large}), the medium chart is the same as the \textit{one line chart} (fig. \ref{fig:icons_medium}) and the small chart is the same as the \textit{single optotype chart} (fig. \ref{fig:tumbling_e_small}). To navigate between the upper and lower chart, the user must swipe the screen up or down and to page between the optotypes on the small chart, the user must swipe left or right.

\begin{figure}[ht!]
\centering
\begin{subfigure}{.5\textwidth}
  \centering
  %\includegraphics[width=.5\linewidth]{images/etdrs_top.png}
  \includegraphics[width=60mm]{images/appgui/icons_medium.png}
  \caption{The medium Icons 1 chart}
  \label{fig:icons_medium}
\end{subfigure}%
\begin{subfigure}{.5\textwidth}
  \centering
  \includegraphics[width=60mm]{images/appgui/tumbling_e_small.png}
  \caption{The small Tumbling E 1 chart}
  \label{fig:tumbling_e_small}
\end{subfigure}
\caption{A medium and small chart}
\label{fig:chart_medium_small}
\end{figure}

The last setting the user can choose is the distance setting. Here the user can enter the distance a patient is from the screen, in centimeters, and then press \textit{Apply Distance}. This has no visible effect on the application but is used to set the optotype sizes on the monitor.

\subsection{Server}
\subsection{Network}

\section{Test Results}
\subsection{Developer walkthrough}
During the developer walkthrough of the system, all charts versions where found and all settings where usable on every chart by using table \ref{tab:test1}. The connection settings where also reachable from everywhere in the application and both connecting and reconnecting was working properly.

\begin{table}[ht!]
\centering
\begin{tabular}{l c c c r}
Chart version		&	Large		&	Medium		&	Small		&	Distance	\\
\hline\hline
ETDRS 1				&	\checkmark	&	\checkmark	&	\checkmark	&	\checkmark	\\
ETDRS 2				&	\checkmark	&	\checkmark	&	\checkmark	&	\checkmark	\\
ETDRS R				&	\checkmark	&	\checkmark	&	\checkmark	&	\checkmark	\\
ETDRS Random		&	\checkmark	&	\checkmark	&	\checkmark	&	\checkmark	\\
\hline
Tumbling E 1		&	\checkmark	&	\checkmark	&	\checkmark	&	\checkmark	\\
Tumbling E Random	&	\checkmark	&	\checkmark	&	\checkmark	&	\checkmark	\\
\hline
HVOT 1				&	\checkmark	&	\checkmark	&	\checkmark	&	\checkmark	\\
HVOT Random			&	\checkmark	&	\checkmark	&	\checkmark	&	\checkmark	\\
\hline
Icons 1				&	\checkmark	&	\checkmark	&	\checkmark	&	\checkmark	\\
Icons Random		&	\checkmark	&	\checkmark	&	\checkmark	&	\checkmark	\\
\hline\hline
					&	Scanning	&	Connecting	& 	Reconnecting	& \\
Connection			&	\checkmark	&	\checkmark	&	\checkmark	&	\\
\hline\hline
\end{tabular}
\caption{Result of the developer walkthrough. \label{tab:test1}}
\end{table}


\subsection{Optotype Sizes}
The optotype sizes were calculated for and measured on a 15.6 inch computer screen and a 37 inch TV, both with 1920x1080 resolution. As the TV is larger than the computer screen, and have the same resolution, each pixel is bigger and thus, the calculated size in pixels is smaller (table. \ref{tab:optotype_test}). The height in centimeters, rounded to two decimals, are both equal to the expected height and the test is considered successful.


\begin{table}[ht!]
\centering
\begin{tabular}{ p{1cm}|p{1cm}p{1cm}|p{1cm}p{1cm}|p{1cm} }
	 &	\multicolumn{2}{c}{\parbox{2cm}{\centering 15.6", 1920x1080}} & \multicolumn{2}{c}{\parbox{2cm}{\centering 37", 1920x1080}}	& 	Expected	\\
\hline
	 row  & px & cm & px & cm & cm \\
\hline\hline				
								1	&	162	&	2.91	&	68	&	2.91	&	2.91\\	
								2	&	128 & 	2.31	& 	54	&	2.31	&	2.31\\
		 						3	&	102 & 	1.84	& 	43	&	1.84	&	1.84\\
								4	&	81 	& 	1.46	& 	34	&	1.46	&	1.46\\
								5	&	64 	& 	1.16	& 	27	&	1.16	&	1.16\\
								6	&	51 	& 	0.92	& 	22	&	0.92	&	0.92\\
								7	&	41 	& 	0.73	& 	17	&	0.73	&	0.73\\
								8	&	32 	& 	0.58	& 	14	&	0.58	&	0.58\\
								9	&	26 	& 	0.46	& 	11	&	0.46	&	0.46\\
								10	&	20 	& 	0.37	& 	9	&	0.37	&	0.37\\
								11	&	16 	& 	0.29	& 	7	&	0.29	&	0.29\\
								12	&	13 	& 	0.23	& 	5	&	0.23	&	0.23\\
								13	&	10 	& 	0.18	& 	4	&	0.18	&	0.18\\
								14	&	8 	& 	0.15	& 	3	&	0.15	&	0.15\\
										\hline
\end{tabular}
\caption{Height of optotypes for a 15.6 and 37 inch screen, both with a 1920x1080 resolution. The rightmost column displays the expected height of the optotypes in cm} \label{tab:optotype_test}
\end{table}


\subsection{Usability}
\subsection{Server Detection}
The server detection test was done 3 times on a stable WiFi network. According to the results (Table. \ref{tab:server_detection_test}), every single one of the tests performed was successful in finding a server over 90\% of the time. The average success rate of the 3 tests was about 96.3\% which means that the server detection requirements are met.

\begin{table}[ht!]
\centering
\begin{tabular}{l c r}
Attempt	&	failures	&	succeses	\\
\hline
1	&	7	&	93	\\
2	&	1	&	99	\\
3	&	3	&	97	\\
\end{tabular}
\caption{The number of failures and successes during the server detection tests. \label{tab:server_detection_test}}
\end{table}

%\begin{figure}[ht!]
%\end{figure}

\section{Bugs and Limitations}
Bugs:
\begin{itemize}
\item Apply button changes to down
\item Logmar duplicate values appear in top left corner of medium chart
\item Distance resets when changing chart types
\end{itemize}


\chapter{Conclusion}

\section{Discussion}

\subsection{Advantages of a digital system}
The advantages of a digital system is the ability to change the layout, optotypes and size of the chart without having to carry away the old chart and then carrying back a new. This will save both time and space as only one screen and a small Android device is necessary. This system will also allow the visual acuity tests to take place in rooms that previously where too small, as the system will display optotypes at a size suitable for the distance the patient is from the monitor.

\subsection{Disadvantages of a digital system}
A disadvantage with this system is that digital screens become very expensive at large sizes, which limits the maximum distance a patient can sit from the screen. Another disadvantage is that the resolution, or rather pixel density, will limit how close to a screen a patient can sit without the optotypes looking pixelated. Pixel density is a measurement of how many pixels (dots) can fit in a small section of the screen, usually denoted as Pixels Per Inch (PPI) or Dots Per Inch (DPI). The higher the DPI is the better the quality of small details will be. This is very important when rendering type fonts because the edges should be as smooth as possible.
\subsubsection{Alternate solutions}

\section{Future Work}
\begin{itemize}
\item grating
\item color blind tests
\item change logmar
\item constraints to random function
\end{itemize}

\bibliography{references}{}
\bibliographystyle{ieeetr}

\newpage
\appendix
\chapter{Testing} \label{App:test}
\section{Developer walkthrough}
\begin{table}[ht!]
\centering
\begin{tabular}{l c c c r}
Chart version		&	Large		&	Medium		&	Small		&	Distance	\\
\hline\hline
ETDRS 1				&		&		&		&		\\
ETDRS 2				&		&		&		&		\\
ETDRS R				&		&		&		&		\\
ETDRS Random		&		&		&		&		\\
\hline
Tumbling E 1		&		&		&		&		\\
Tumbling E Random	&		&		&		&		\\
\hline
HVOT 1				&		&		&		&		\\
HVOT Random			&		&		&		&		\\
\hline
Icons 1				&		&		&		&		\\
Icons Random		&		&		&		&		\\
\hline\hline
			&	Scanning	&	Connecting	& Reconnecting	& \\
Connection	&	&	&	& \\
\hline\hline
\end{tabular}
%\label{tab:test1_empty}
\caption{\label{tab:test1_empty}}
\end{table}


\section{Usability test procedure}
\begin{enumerate}\bfseries

\item Start application

\item Connect to server

\item Display ETDRS 1 Large upper chart with a distance of 2 meters

\item Display ETDRS 1 Large lower chart with a distance of 2 meters

\item Display HVOT Random chart medium with a distance of 1.7 meters

\item Generate a new HVOT chart

\item Display Icons 1 small chart with a distance of 1.5 meters

\item Page through Icons 1 small chart with a distance of 1.5 meters

\item Reconnect to server

Best thing with GUI\\
bra att meny är framme hela tiden

Needs improvement\\
click to connect not obvious

\end{enumerate}
\end{document}
