\documentclass[12pt,a4paper,notitlepage]{report}
\usepackage[utf8]{inputenc}
\usepackage{cite}
\usepackage{etoolbox}
\usepackage{listings}
\usepackage{graphicx}
\usepackage{subcaption}
\usepackage{amsmath}
\usepackage{titlesec}
\usepackage[nottoc, notlof]{tocbibind}
\usepackage{pgfplots}
\usepackage{siunitx}
\usepackage{multirow}
% might need later, for diagrams
%\usepackage{tikz}
%\usetikzlibrary{arrows,shadows}
%\usepackage{pgf-umlsd}

\def\checkmark{\tikz\fill[scale=0.4](0,.35) -- (.25,0) -- (1,.7) -- (.25,.15) -- cycle;}

%%%%%%%% Bibliografi %%%%%%%%%%%%%%%%%%%%%%%%%%%%%%%%%
%% Ändrar namnet på referens headern
\renewcommand\bibname{ References}
%%%%%%%%%%%%%%%%%%%%%%%%%%%%%%%%%%%%%%%%%%%%%%%%%%%%%%

%\titleformat{\chapter}[display]
%  {\normalfont\bfseries}{}{0pt}{\Large}

\titleformat{\chapter}[display]
 {\normalfont\bfseries}{}{0pt}{\Large \thechapter }

%%%%%%%%%%%%%%%%% Start document %%%%%%%%%%%%%%%%%%%%%
\begin{document}
\pagenumbering{roman}

\title{Digitalization of Visual Acuity Testing}
\author{Anders Ohlsson\\Christian Johansson\\Svante Nilsson}
\maketitle

\begin{abstract}
Digitalization of Visual Acuity Testing is a project about digitally displaying and manipulating optician's eye charts. The charts are displayed on a digital screen with a lightweight server running on a Raspberry Pi and controlled over a wireless network via an Android tablet. 
%%This solves several problems related to the distance from the patient to the chart, while also providing more ways for the optician to tailor the examination to the patient.
This allows the charts to adapt when placed at different distances from the patient, while also allowing the optometrist to tailor the examination to the patient by switching between multiple sheet designs and specialized sets of characters. 
The Digitalization of Visual Acuity Testing project has improved upon a proof-of-concept system that has been used at the Uppsala University Hospital to test whether or not digital screens are a viable alternative to physical charts. The project fulfils its goals and has performed well under testing, but more testing is needed to make sure it meets the demands of practising opticians.
\end{abstract}
\renewcommand{\abstractname}{Sammanfattning}
\begin{abstract}
Digitalization of Visual Acuity Testing är ett projekt om att visa och manipulera syntavlor digitalt. Syntavlorna visas på en bildskärm kopplad till en enkel server som är installerad på en Raspberry Pi och kan styras över ett trådlöst nätverk med hjälp av en Androidsurfplatta. Eftersom tavlorna är digitala kan de anpassas till avståndet mellan patienten och tavlan, samtidigt som de låter en optiker anpassa synundersökningen till patienten genom att tillåta snabbt byte av syntavlans layout och symboler. Projektet bygger på ett testsystem som har använts vid Akademiska Sjukhuset för att undersöka ifall digitala skärmar är ett passande alternativ till tryckta syntavlor. Digitalization of Visual Acuity Testing har uppfyllt sina mål och har presterat väl i tester, men mer tester måste genomföras för att se till att det uppfyller alla krav en optiker skulle kunna ha på projektet.
\end{abstract}
\renewcommand\thechapter{ }
\tableofcontents*
\listoffigures
\clearpage

\renewcommand\thechapter{\arabic{chapter}}
\pagenumbering{arabic}
\setcounter{page}{1}
\chapter{ Introduction}
%\section{Background}
The standard way for an optician to measure a patients visual acuity is to use charts printed with rows of continuously smaller symbols called optotypes~(fig.~\ref{fig:optotypes_example1}), most often normal Latin letters. The patient reads the symbols on the chart in the order specified by the optician until he or she can no longer correctly identify them. Each row on the chart has a pre-calculated number printed beside it, which lets the optician determine which lenses the patient should use.

\begin{figure}[ht!]
\centering
\begin{subfigure}{.3\textwidth}
  \centering
  \includegraphics[width=.4\linewidth]{images/landholt_c_optotype.png}
  \caption{Landholt C}
\end{subfigure}%
\begin{subfigure}{.3\textwidth}
  \centering
  \includegraphics[width=.4\linewidth]{images/tumbling_e_optotype.png}
  \caption{Tumbling E}
\end{subfigure}
\begin{subfigure}{.3\textwidth}
  \centering
  \includegraphics[width=.4\linewidth]{images/lea_optotype.png}
  \caption{LEA}
\end{subfigure}
\caption{Some example optotypes}
\label{fig:optotypes_example1}
\end{figure}

These numbers have been calculated from a set of standardized conditions~\cite{Bailey}. For example there must be adequate lighting in the room so the patient can distinguish the colors and read the letters, but more importantly the charts are meant to be placed a fixed distance from the patient. For most charts, this distance is six meters. 

These conditions cannot always be met. Examination rooms are not always large enough to take measurements from six meters away for example. In those cases, the reference numbers on the sheet are no longer accurate and the optician has to compensate manually~\cite{PGSoderbergOral}. If the standardized conditions cannot be met it can lead to irregularities or faulty measurements in some tests.

There are also several different variations on these eye charts. Different sets of optotypes, ranging from different subsets of Latin letters to more abstract symbols, have been developed for different types of patients, such as those who are analphabetic, or for small children. The layout of these charts also have variations, from the classic Snellen chart~(fig.~\ref{fig:snellen_chart1}) that was developed in the 19th century, to the more modern ETDRS (Early Treatment Diabetic Retinopathy Study) chart~(fig.~\ref{fig:etdrs_chart1}).


In a field so dependent on visual aids and equipment, it is easy to see that there are several possibilities for improvement and expansion. Modern technology can be used to improve these tests, and digitalization can simplify the way opticians work with diagrams.

\begin{figure}[ht!]
\centering
\includegraphics[width=110mm]{images/Snellen_chart.png}
\caption[The Snellen chart]{The Snellen chart (Source: http://en.wikipedia.org/wiki/Snellen chart\cite{img_snellen}).} \label{fig:snellen_chart1}
\end{figure} 

\begin{figure}[ht!]
\centering
\includegraphics[width=120mm]{images/etdrs_chart.jpg}
\caption[The ETDRS chart]{The ETDRS chart, used with permission from Precision Vision \\ (Source: http://precision-vision.com/)\label{fig:etdrs_chart1}}
\end{figure} 

\chapter{ Problem Description}
\section{Problem}
Having efficient and accurate systems is important in a medical environment, especially in a medical environment that handles tests where inaccurate or false results in some cases can cause accidents. Therefore it was important to have a well-built system that mitigates these problems. With the help of professor P.G. Söderberg who is an expert in the field of ophthalmology we have created a system that acts as a prototype for further future work. 

Determining visual acuity is a very labour- and equipment-intensive process, and this workload is also increasing drastically with time. Physical charts are often large and cumbersome, and must be switched if the patient needs a chart with another selection of optotypes or a different layout. Since they are also designed for standardized distances and testing conditions, an optician needs to manually compensate for any deviations from the standard. More and more people are experiencing eye problems, at all ages and in many cultures~\cite{vision_loss}. With more people needing to test their visual acuity, faster and more efficient methods needs to be used.

Accuracy is not the only important aspect to think of when implementing a system. It is also important to think about expanding and improving on how the system can be used. To introduce new functionality that makes the life of the system's user a lot more simple and efficient is one of those important improvements. For example there are problems with using the standardized charts, such as older patients getting lost among the columns and rows, and other patients memorizing the rows of letters in an attempt to cheat. Giving the optician the option of using randomly generated charts and easily switching between specialized chart layouts can greatly increase the speed and accuracy of testing.

These are a few of the aspects that we took into consideration as we built our system. A simple system that implements digital eye charts had already been built by professor P.G. Söderberg, though that system is not as sophisticated or expandable as needed. We based some of our system's content upon how the original system was built. Mostly this system used the layouts of the charts already created since those charts have been shown to work. 

Using the original system as a basis for our own we built a system that

\begin{itemize}
	\item Has an application for the Android tablet with an improved user interface, both aesthetically and functionally.
	\item Controls a server that renders the eye charts on a screen over WiFi using the Android tablet.
	\item Can alter the scale of the rendered symbols so that the patient can be closer or further away from the screen.
	\item Has functionality to tailor the testing process by, among other things, changing the sets of symbols displayed or removing clutter from the charts.
\end{itemize}

%\section{Motivation}
%A few things can be found in the working environment of the optician. Determining visual acuity is a very labour- and equipment-intensive process. It is a lot of manual labour with much time wasted when setting up and changing the equipment used. This work-load is also increasing drastically with time. More and more people are experiencing eye problems, at all ages and in many cultures. \cite{vision_loss} With more people needing to test their visual acuity, faster and more efficient methods needs to be used. 

%Another fact is that physical eye charts are not as accurate as they should be for accurate measurements. They are also big in size and static, static in this context because they are fixed and cannot be changed. Because the handling of printed charts is slow and not as accurate as needed there exists a need for improvement. 

%A digital eye chart has increased functionality, accuracy and efficiency which are all important aspects to approve upon. A digitalization also gives the user a greater possibility for extending the usage of the system.

\section{Issues}
There were several technical problems that we needed to solve. We needed to find a suitable computer system that would act as the server, its job being storing and presenting the desired images to the screen. Our choice was to use a Raspberry Pi, since it is a cheap and compact alternative to a full-size computer that is more than powerful enough for our purposes. Although since the Raspberry Pi's CPU is running on an ARM-architecture there were a few compatibility problems. For example, the newly redesigned Java graphics library JavaFX, which is eventually going to replace the older Swing library, was only partially supported on the ARM CPU-architecture. Not being able to use the latest features had to be considered since it could create problems with future Java updates when some things get deprecated. If Java Swing would become deprecated in the future it should not be too difficult to transfer the code to a compatible Java FX version.

Designing the Android app for the tablet also brought technical problems, both from the limitations of portable computers, and in making our software adaptable enough to work on as many brands of tablets as possible. For example, the symbols we display are stored in the .SVG (Scalable Vector Graphics) format, which was not natively supported on Android. Because of this we used a licensed third-party software library to render them. 

Finally, we had the issue of networking. Our concern was being able to reliably control the server over a wireless network while having the response be fast and accurate. Any connection problems between the controller and the server would make actions seem slow, inconsistent or unresponsive. This is true both for mobile networks and wireless networks, like the ones we are using. This was an equally important problem compared to the other ones, but network communication is a well-explored field, and support for the small amount of communication we need to do between the server and the tablet exists natively in Java.

%\section{Ethical considerations}
%Our ethical obligations with this project is providing a functional and stable product. If our system does not perform to specification, a lot of patients could get erroneous measurements during testing, costing a lot of time and money to correct. This is very important to think about since we are dealing with treatment of people. All of this has to be considered even if the final results will not be used clinically.

%\chapter{ Problem Description}
%\section{Problem}
%\subsection{Gammalt, bakas in i det ovan (Ta bort denna subsection)}
%The problem with today's visual acuity testing charts are that they are very clumsy to work with. The charts are printed and contains only one static selection of symbols, also called optotypes. This could be optotypes for people with the ability to read or an 'E' shape rotated in different directions which allows children who are unable to read, or illiterate people, to identify which symbol they see. These static charts are also designed to be viewed from one specific distance. This causes problems when dealing with environments that are too small, for example erroneous measurements. 
%There are several weaknesses with visual acuity testing today that we have worked on improving. Physical charts are often large and cumbersome to switch if the patient needs a chart with another type of optotypes. Since they are also designed for standardized distances and testing conditions, an optician needs to manually compensate for any deviations from the standard. 

%There are also other problems with using standardized charts, such as older patients getting lost among the columns and rows, and other patients memorizing the rows of letters to cheat. Giving the optician the option of using randomly generated charts and easily switching between specialized chart layouts can greatly increase the speed and accuracy of testing.

%Another problem we faced is designing a user interface that can give the optician easy access to these specialized charts and layouts. An intuitive design so that it is easy to learn, and a user friendly design so that users don't accidentally make unintentional or irreversible changes.

%\subsection{Nytt, behålls, bara denna rubrik tas bort när det gamla är fixat \\ se över den här texten så det inte blir nya problem}
%We created a system that allows opticians to quickly prepare and perform an eye examination using a chart displayed on a monitor controlled by an Android device. The system must
%----------------SKRIV MER HÄR!!!!!!!!!----------- (ELLER SKRIV OM!!!!)

\section{Delimitations}
Due to limits in time and manpower, this project was not intended to result in a complete working system to be used clinically, but rather a working prototype that can be used to evaluate digital visual acuity charts. To complete the project on time, as much as possible of Java's and Android's built-in libraries was used in order to avoid recreating code that already existed, thus saving time. Neither Java nor Android include libraries for rendering SVG-files so third party libraries was used instead.

We did not evaluate a digital screen's effect on the eye compared to a physical chart nor did we analyze why a chart should have the layouts it has. As the visual acuity charts are made after a very strict specification, no new chart layouts were made. Any new chart we would have created would be considered useless without backing research and they could be misleading and misused by future opticians that in the end, could result in patients getting an inaccurate result.

No security features, such as encrypted network traffic or protection against several tablets connecting to the same server at the same time, was implemented. Instead, it is left to the optician or whoever owns the hardware to make sure that a secure network is used. Since there is no patient data or other private information being transmitted or stored we did not see the need for secure connections.

No automatic update system was included, instead the hardware owner has to manually update both the Android application and the server. Automatic updating was not a core feature, and could therefore be saved for future work.

\section{Requirements}
%For this project to be considered successful, the system must be able to display the chart chosen by the user. This means that the user have to be able to control which chart type to display on the monitor from an Android device. The types of charts that should be included are the four Axanivis charts, the upper, the lower, the one line and the single optotype chart. The user also has to be able to choose if the current chart should use the Sloan, Tumbling E, HVOT or LEA optotypes.

For this project to be considered successful, the system had to be able to display the chart chosen by the optician. This means that the optician had to be able to control which chart type to display on the monitor from an Android device. The types of charts that were included are the four Axanivis charts (fig. \ref{fig:axanivis}), which comes from a project with the same name under development by P.G. Söderberg \cite{PGSoderbergOral}. The Axanivis charts are based on the ETDRS chart but are designed to be displayed on a digital screen instead of being printed on a physical chart. 

\begin{figure}[ht!]
\centering
\begin{subfigure}{.5\textwidth}
  \centering
  %\includegraphics[width=.5\linewidth]{images/etdrs_top.png}
  \includegraphics[width=50mm]{images/etdrs_top.png}
  \caption{Axanivis upper chart, based on the first five rows of the ETDRS chart.}
  \label{fig:etdrs_upper1}
\end{subfigure}%
\begin{subfigure}{.5\textwidth}
  \centering
  \includegraphics[width=50mm]{images/etdrs_bottom.png}
  \caption{Axanivis lower chart, based on the last nine rows of the ETDRS chart.}
  \label{fig:etdrs_lower1}
\end{subfigure}
\begin{subfigure}{.4\textwidth}
	\centering
	\includegraphics[width=50mm]{images/etdrs_one_line.png}
  \caption{Axanivis one line chart, based on a single column of the ETDRS chart that is turned horizontal.}
\end{subfigure}
\begin{subfigure}{.4\textwidth}
	\centering
	\includegraphics[width=50mm]{images/etdrs_single.png}
  \caption{Axanivis single optotype chart. Displays a single optotype at a time.}
\end{subfigure}
\caption{The four Axanvis charts}
\label{fig:axanivis}
\end{figure}

The optician also had to be able to choose if the current chart should use the Sloan, Tumbling E, HVOT or LEA optotypes (fig. \ref{fig:optotypes_example1}). The Sloan and HVOT optoypes consists of a different sets of letters, the Tumbling E consists of the letter E rotated in different angles and LEA is a set of custom symbols. \cite{Colenbrander}

\begin{figure}[ht!]
\centering
\begin{subfigure}{.3\textwidth}
  \centering
  \includegraphics[width=.4\linewidth]{images/landholt_c_optotype.png}
  \caption{A Sloan C}
  \label{fig:landholt_c1}
\end{subfigure}%
\begin{subfigure}{.3\textwidth}
  \centering
  \includegraphics[width=.4\linewidth]{images/tumbling_e_optotype.png}
  \caption{A Tumbling E}
  \label{fig:tumbling_e1}
\end{subfigure}
\begin{subfigure}{.3\textwidth}
  \centering
  \includegraphics[width=.4\linewidth]{images/lea_optotype.png}
  \caption{A LEA House}
  \label{fig:lea1}
\end{subfigure}
\caption{Some example optotypes}
\label{fig:optotypes_example1}
\end{figure}

%It is very important that the chart on the monitor displays all optotypes with the correct sizes and positions. In order to do this, the user must to be able control the distance from which the patient is watching the monitor. It is also very important that the correct sizes, in LogMAR (Logarithmic of the Minimum Angle of Resolution), which is a logarithmic unit of measurement dependant on the view angle of a patient \cite{Bailey}, is displayed on the Android device.

It was very important that the chart on the monitor displays all optotypes with the correct sizes and positions. In order to do this, the optician had to be able to input the distance from which the patient is watching the monitor. The sizes of the the optotypes are measured in LogMAR (Logarithmic of the Minimum Angle of Resolution), which is a logarithmic unit of measurement dependant on the view angle of a patient \cite{Bailey}. This means that the optotypes are smaller when a patient sits close to the monitor and larger when a patient sits far away. It also allows the optician to use the same LogMAR values during visual acuity tests no matter the distance the patient view the monitor.

The intended user for this system is an optician, and should have all the necessary knowledge and education on how to perform an eye examination. The optician should not need any more than basic knowledge about an Android device, such as being able to start the device, connect to a wireless network, start an application and how to perform simple Android gestures. This does not mean that it has to be easy to install and set up, though it is a goal for future expansion. 

The Android application follows the Android design principles set up by Google \cite{android_design}. This made the application behave similar to other Android applications which results in users already familiar with Android feel familiar to this application as well.

The system allows for automatic server detection so that opticians do not have to manually connect by entering IP addresses. This allows for a quick and easy way to connect to a server. In order for the server detection to be considered working properly, it had to detect a server on a stable network at least 90\% of the time. With a success rate of 90\%, it was very uncommon to fail the detection twice in a row, while keeping the scanning time relatively low.

As this project only aimed to complete a working prototype of the system, it was very important that it was easy to continue the development and add more features without rewriting large parts of the code.

\section{Evaluation \label{sec:evaluation}}
In order to evaluate if all requirements had been met, the system underwent four tests. The goal of the first test was to make sure that the optician actually could choose a specific chart with specific optotypes to be viewed from a specific distance. This was done by the developers going through all charts, with all types of optotypes and setting the distance for each one of them. Table \ref{tab:test1_empty} was used to achieve this by systematically walk through every setting.

The second test made sure that the optotypes had the correct sizes on the monitor. This test was done by calculating the expected sizes of the optotypes in centimeters for different screen sizes with the help of Matlab. The calculated sizes was then compared to the sizes output by the system. If there was a match, the optotypes was measured by a physical measurement tool such as a ruler or Vernier scale in order to make sure that they had the calculated size when they were displayed on the monitor. The physical measuring was required to make sure that the system worked as intended in practice and not only in theory.

The third test determined how user friendly the system was. This was done by letting a test user with no previous knowledge of the system, attempt to connect to a server and perform some basic tasks like setting a specific chart with a specific optotype or setting a specific distance (App. \ref{app:usability}). If the test user testing had no previous knowledge of visual acuity charts, a brief explanation was given so that the test did not fail because the test user, for example, did not know what "HVOT" was. Once the test was completed, the test user provided feedback pointing out possible faults in our design. This feedback was then used to improve the design. If the test user performd the entire test without help, the test was considered successful.

The last test was a quantitative test to see how well the server detection performed. The test involved the Android application trying to detect a server on a stable WiFi a hundred times and see how many times the server was detected. This test was done several times and the average number of detections in a hundred tries determined its success rate. Testing the system on anything other than a stable WiFi was unnecessary as it was only intended to work on a stable local network. If the optician's tablet could not connect to their local WiFi reliably there might have been problems, but the source of those problems were outside of our project, and we could not affect it.

\section{Related work}

\subsection{Visual acuity charts}
%Several different versions of visual acuity chart exist today. The ETDRS (Early Treatment Diabetic Retinopathy Study) chart~(fig.~\ref{fig:etdrs_chart1}) is the most commonly used today and is an improvement to the older \textit{Snellen chart}(fig. \ref{fig:snellen_chart1}, page \pageref{fig:snellen_chart1}). The ETDRS chart was designed and developed by Ferris et al \cite{Ferris}, based on the work of Bailey and Lovie \cite{Bailey}. 

Several different versions of visual acuity chart exist today. The ETDRS (Early Treatment Diabetic Retinopathy Study) chart~(fig.~\ref{fig:etdrs_chart1}, page \pageref{fig:etdrs_chart1}) is used at Uppsala Univeristy Hospital today and many other places around the world, and is an improvement to the older \textit{Snellen chart}(fig. \ref{fig:snellen_chart1}, page \pageref{fig:snellen_chart1}). The ETDRS chart was designed and developed by Ferris et al \cite{Ferris}, based on the work of Bailey and Lovie \cite{Bailey}. The Snellen chart design is the most used design globally, but the ETDRS design has been rising in popularity since it has an increased testing accuracy because of its more legible font and because it has the same number of letters on each row.

%\begin{figure}[ht!]
%\centering
%\includegraphics[width=60mm]{images/Snellen_chart.png}
%\caption{The Snellen chart (Source: http://en.wikipedia.org/wiki/Snellen chart\cite{img_snellen}).} \label{fig:snellen_chart}
%\end{figure} 


\subsection{Digital visual acuity charts}
%Digital Eye Chart \cite{digitaleyechart} and iChartPlus \cite{ichartplus} are two applications supposed to display a visual acuity chart on a digital screen. Both applications are computer applications, which requires the computer to be connected to the screen. The companies behind the products provide very little information and the exact specifications of the charts are not available.

There are several other commercial applications that display visual acuity charts on digital screens, for example Digital Eye Chart \cite{digitaleyechart} and iChartPlus \cite{ichartplus}. Most of these alternatives are computer applications which require the computer it runs on to be connected to the screen the chart is displayed on. Both Digital Eye Chart and iCharPlus, like many other of the commercial applications, provide a remote control that can switch charts and other features. The differences between these systems and this project are in using a tablet computer as a remote, allowing a visual interface with touch controls that connects with the screen and server wirelessly instead of with a video cable. A visual interface allows more types of interactions than a remote control and also allows us to show values and information on the tablet held by the optician instead of on the screen, removing clutter that might distract the patient.

\subsection{Media streaming devices}
Several media streaming devices exists that allow digital content to be received and displayed on a connected screen. Examples of such devices are Chromecast \cite{chromecast} and Apple TV \cite{appletv} which you can plug into a TV and allows you to view images or watch movies that are stored on your phone or computer. Both applications also allows developers to develop their own applications to display custom content but requires the developer to enter an agreement with respective company. Both these services allow you to mirror the screen of your tablet or phone to the TV screen, and could be used to display an eye chart. However, unless the screen on your tablet and the screen on your TV have the same resolution the image is be stretched or resized, reducing the quality of the image. If the screen is mirrored there is also a lot of network traffic needed to stream the screen over to the Chromecast or Apple TV, putting a strain on the tablet's battery life. The server in our project's solution displays the charts which are stored  on the server, minimizing the need for network traffic while working at the screens resolution without stretching or resizing.


%There exists today several companies that have taken on the task of digitalizing the usage of eye charts.
%
%\subsubsection{Chromecast and Apple TV}
%The Chromecast \cite{chromecast} and Apple TV \cite{appletv} are small devices which are plugged into a TV and enables mobile phones and tablets to stream media to them. The devices are small and easy to handle and a lot of applications support streaming to them. The Chromecast is specifically made for Android devices and Apple TV for iOS device, but they both support each other.
%
%\subsubsection{Digital Eye Chart}
% Digital Eye Chart \cite{digitaleyechart} is a visual acuity testing program that attempts to solve the problem of having several charts by using a monitor instead. The user can then, with the help of a computer, change the symbols on the chart and even generate a random chart. 
%
%\subsubsection{iChart+}
%iChartPlus \cite{ichartplus} is another program which is similar to Digital Eye Chart but also have an adjustable viewing distance. 
%
%\subsubsection{AxAnIvIs}
%AxAnIvIs is a prototype that is currently under testing at Uppsala University Hospital. The project attempts to improve on how we perform visual acuity tests by digitalizing them. It adds more types of charts than the standard physical ones that exists today. The system uses a webbrowser running on an Android tablet that is connected to a screen. A library of vector graphic files are displayed in the webbrowser. These files contains the different charts as well as some other specialized tools. Currently the Android tablet has to be held in a fixed position in the examination room, restrained by the cable going to the screen. The charts used are fixed and not very easily changed, which makes it very hard to try out new techniques and methods of measuring.
%
%\subsubsection{Our project}
%Most of the before mentioned systems have some form of negative side that we try to improve on, mostly problems about high cost and difficulties with extending the software. We are different with that we want to create a system that is cheap, easy to understand and easy to maintain. Creating a system that have all the previously mentioned sides is one of the goals we have. Another important goal is creating a system that really is speacialized on measurements of visual acuity. All these points are important to give the users the best experience.

\chapter{ Visual Acuity Testing}
\section{Visual Acuity}
The measurement of a persons visual acuity has an important constraint that needs to be followed. This constraint is the fact that our eyes works best physiologically from greater distances. On the other hand, at shorter distances our eyes tend to make what we see look unfocused. To make up for this shift in focus a minimum testing distance has to be enforced to ensure that the shift in focus stays as small as possible. This distance has by scientific research been set to be six meters. \cite{Acuity_Book}

\subsection{Measurement distance}
The specific distance of six meters is derived from how our eyes bend light. From that distance or greater the rays of light entering our eyes are nearly parallel to each other, parallel enough to make it unnoticeable. It is important that the rays of light that hit the eyes are parallel because then the light hits a more focused area in the back of the eyes. This happens because of the convex shape of the eye's lens. A convex shaped lens focuses parallel light onto the focal point. Since a convex lens focuses parallel light to a focal point a concave lens instead spreads light coming from a focal point to parallel rays. Combining these two type of lenses we can create a more focused light. \cite{Optics_Book}

\subsection{Defocus correction}
To make up for the shift in focus a suitable lens is put in front of the eye at about the distance that you would wear glasses. The type of lens mostly depends on the shape of your eye, any eye disease you might have or the distance from the lens to the eye. An important property of the lenses is at what distance the focal point lays at. So, to create a more focused light, which is exactly what glasses do, we have to put a lens in front of the eye to create more parallel light. These lenses can either be plus or minus lenses, either spreading or contracting the light. \cite{Optics_Book}

In optician's terms a minus glass is a lens that focuses light coming further away (subtraction) and a plus glass is a lens that focuses light from nearer (addition). A real world example of this is glasses correcting ~near- and farsightedness. \cite{Acuity_Book}

\subsection{Angular magnification}
% TODO: Explain using formulas how the angular magnification works
\begin{figure}[h]
\centering
\includegraphics[width=120mm]{images/Angular_magnification.png}
\caption{Angular magnification at short distances\label{angular}}
\end{figure}

Though using lenses is a way of corrected for testing at short distances, a few problems arise from this. Figure \ref{angular} shows the problem of angular magnification. The magnification is a problem in visual acuity testing because if the optotypes on the eye chart are enlarged then an error shows up in the results.

\newpage
\begin{figure}[h]
\centering
\begin{tabular}{| l | l | l | l |}
    \hline
    Distance to chart (m) & Addition (D) & Magnification (Rel.) & Error (logMar) \\ \hline
    4                     & 0.25         & 1.5                  & 0.17           \\ \hline
    2                     & 0.5          & 3.5                  & 0.48           \\ \hline
    1                     & 1            & 6.0                  & 0.78           \\ 
    \hline
    \end{tabular}
    \caption{Magnification due to addition for closer distance\label{magtable}}
\end{figure}

There is need to have a standard to follow on how much addition or subtraction needed for a specific distance. Figure \ref{magtable} shows what values to use when testing visual acuity at shorter distances than six meters. \cite{PGSoderbergOral}

\section{Chart Design}
The visual acuity charts used in this project are based on the ETDRS (fig. \ref{fig:etdrs_chart1}, page \pageref{fig:etdrs_chart1}) \cite{Ferris} and the Axanivis (fig. \ref{fig:axanivis}) \cite{PGSoderbergOral} charts developed by P.G. Söderberg at Uppsala University Hospital. 

%\begin{figure}[ht!]
%\centering
%\includegraphics[width=100mm]{images/etdrs_chart.jpg}
%\caption{Example of a ETDRS chart used with permission from Precision Vision \\ (Source: http://precision-vision.com/)\label{fig:etdrs_chart}}
%\end{figure} 

\subsection{Optotypes} \label{Optotypes}
An optotype is a symbol that is used on a visual acuity chart. The optotypes are all of the same width and height and placed on a five by five grid. The most common type of symbol to use are letters from the Latin alphabet but can be different depending on the patient's needs\cite{Colenbrander}.

\textit{Sloan letters} are defined to be the ten optotypes \textit{C, D, H, K, N, O, R, S, V and Z}. They have been chosen because they are easily differentiated from each other \cite{Ferris} and are used by most visual acuity charts\cite{Colenbrander}.

\textit{Landholt C} is a sloan \textit{C} that is rotated by 0, 90, 180 or 270 degrees and is usually used in a scientific environment. The Landholt C is used as reference when deciding the how difficult it is to read a specific optotype\cite{Colenbrander}.

\textit{Tumbling E} is like the Landholt C, but with the optotype \textit{E} instead of \textit{C}. This optotype can be used instead of the Sloan collection of optotypes if the patient doesn't know how to read. The patient can instead show or tell the optician in what direction the \textit{E} is rotated\cite{Colenbrander}.

\textit{HVOT} are the letters \textit{H},\textit{V},\textit{O} and \textit{T}. These letters are chosen because they look the same if they are mirrored on the horizontal axis.

\textit{LEA} are four symbols that look like a house, circle, box and apple. These symbols similar to HVOT since they look the same if you mirror them on the horizontal axis.

\begin{figure}[ht!]
\centering
\begin{subfigure}{.3\textwidth}
  \centering
  \includegraphics[width=.4\linewidth]{images/landholt_c_optotype.png}
  \caption{Landholt C}
  \label{fig:landholt_c}
\end{subfigure}%
\begin{subfigure}{.3\textwidth}
  \centering
  \includegraphics[width=.4\linewidth]{images/tumbling_e_optotype.png}
  \caption{Tumbling E}
  \label{fig:tumbling_e}
\end{subfigure}
\begin{subfigure}{.3\textwidth}
  \centering
  \includegraphics[width=.4\linewidth]{images/lea_optotype.png}
  \caption{LEA}
  \label{fig:lea}
\end{subfigure}
\caption{Some example optotypes}
\label{fig:optotypes_example}
\end{figure}

\pagebreak
\subsection{ETDRS Chart}
The ETDRS chart consists of fourteen rows with five optotypes on each row. The size of each optotype is measured in LogMAR which is the base 10 logarithm of the angle of one fifth of an optotype (eq. \ref{eq:logmar} and fig. \ref{fig:logmar_calculation}) \cite{Bailey}. The angle unit is measured in arcminutes, where one arcminute is $\dfrac{1}{60}$ degrees or $\dfrac{\pi}{10800}$ radians. If the angle v in figure \ref{fig:logmar_calculation} is equal to 8 arcminutes, then the size of the letter would be $log_{10}(8) \approx 0.903\ LogMAR$.

%\begin{figure}[h]
%\centering
%\includegraphics[width=100mm]{images/etdrs_chart.jpg}
%\caption{Example of a ETDRS chart \\ (Source: http://precision-vision.com/)%\label{fig:etdrs_chart}}
%\end{figure} 

\begin{equation}
	\begin{split}
  		LogMAR & = log_{10}(angle)
  	\end{split}
  	\label{eq:logmar}
\end{equation}

\begin{figure}[ht!]
\centering
\includegraphics[width=110mm]{images/logmar_calculation.png}
\caption[LogMAR calculation]{The angle v is used to calculate the size in LogMAR of the optotype.(Source: Christian Johansson)\label{fig:logmar_calculation}}
\end{figure} 

The optotypes on the top row of the chart is of the size 1.0 LogMAR and then each row below is decremented by 0.1 LogMAR to the last row which is -0.3 LogMAR. This means that each optotype is approximately 1.2589 the height of the optotypes on the row below (fig. \ref{fig:chart_size_plot}) \cite{Ferris}. The spacing between each optotype is equal to the width of an optotype from the same row and the spacing between each row is equal to the height of an optotype on the lower row (fig. \ref{fig:etdrs_chart_sizes}).

\begin{figure}[ht!]
\centering
\includegraphics[width=110mm]{images/etdrs_chart_sizes.png}
\caption{The sizes and positions of the first three optotypes of row $n$ and $n+1$ where $x_n \approx 1.2589 x_{n+1}$. \label{fig:etdrs_chart_sizes}}
%\caption{The gray squares represents optotypes that on the upper row has a width and height of $x_n$ and on the lower row $x_{n+1}$.  \label{fig:etdrs_chart_sizes}}
\end{figure} 

\begin{figure}[ht!]
\begin{tikzpicture}
    \begin{axis}[
      xmin=1,xmax=14,
      ymin=0,ymax=180,
      xlabel={row},
      xlabel near ticks,
      ylabel={pixels},
      ylabel near ticks
    ]
      \addplot[color=blue, mark=*, only marks] plot coordinates{(1,162)(2,128)(3,102)(4,81)(5,64)(6,51)(7,41)(8,32)(9,26)(10,20)(11,16)(12,13)(13,10)(14,8)};
    \end{axis}
\end{tikzpicture}
\caption[Optotype Size Graph]{The height of optotype from 2 meters on a 15.6" screen with a resolution of 1920x1080 pixels. (Source: Christian Johansson)\label{fig:chart_size_plot}}
\end{figure}

%\newpage
\subsection{Axanivis Charts}
The Axanivis charts are an extension to the ETDRS chart that modifies the original chart and adds two new chart layouts and digitalizes them so they can be displayed on a screen. Since the original ETDRS chart is made for a physical chart, it doesn't have an aspect ratio \cite{Ferris} that is close to a common monitor. To solve this, the original chart is split into an upper (fig. \ref{fig:etdrs_upper}) and a lower chart (fig. \ref{fig:etdrs_lower}). The upper chart consists of the five first rows and the lower chart consists of the the remaining nine.

\begin{figure}[ht!]
\centering
\begin{subfigure}{.8\textwidth}
  \centering
  %\includegraphics[width=.5\linewidth]{images/etdrs_top.png}
  \includegraphics[width=100mm]{images/etdrs_top.png}
  \caption{Upper Chart}
  \label{fig:etdrs_upper}
\end{subfigure}%

\begin{subfigure}{.8\textwidth}
  \centering
  \includegraphics[width=100mm]{images/etdrs_bottom.png}
  \caption{Lower Chart}
  \label{fig:etdrs_lower}
\end{subfigure}
\caption[Upper and lower ETDRS charts]{The upper and lower ETDRS charts (Source: P.G. Söderberg)}
\label{fig:etdrs_upper_lower}
\end{figure}

The two new chart layouts used is constructed using the same principles, like height and spacing of optotypes, as the original ETDRS chart. But instead of having 14 rows of 5 optotypes each, the first new layout takes a single column of the ETDRS chart and rotates it 90 degrees (fig. \ref{fig:etdrs_one_line}) so it is all on one line. This makes it wider than it is high which suits a monitor better. This also allows the patient to read from left to right, which is the common way to read in Europe and America. This makes it easier and feel and more natural to read for a patient used to read in English or any other European language \cite{PGSoderbergOral}. Since the patient only has to read one letter of each size, this layout gives a quicker but more inaccurate result. This is useful to get a quick estimate on the visual acuity of the patient.

\begin{figure}[ht!]
\centering
\includegraphics[width=100mm]{images/etdrs_one_line.png}
\caption[One-line eye chart]{The one line chart (Source: P.G. Söderberg)}
\label{fig:etdrs_one_line}
\end{figure}

\pagebreak
Once an estimated visual acuity has been determined from the one line chart, the optician can select an optotype and show the second new layout, which only displays one optotype at a time (fig. \ref{fig:etdrs_single}). This layout allows the patient to easily focus on a single optotype at a time. The optician can then page between the different optotype sizes in order to determine the visual acuity.

\begin{figure}[ht!  ]
\centering
\includegraphics[width=90mm]{images/etdrs_single.png}
\caption[Single optotype chart]{The single optotype chart where the red circle indicates the area the patient should focus on. Note that this is the view on the optician's screen, and that the screen displayed to the patient contains only the ring and the optotype.}
\label{fig:etdrs_single}
\end{figure}

\chapter{ Implementation}
\section{Methods}
\subsection{System structure}
The system is divided into three major parts, an Android application, a server and networking. The Android application and server is modelled after the MVC (Model-View-Controller) pattern, where the model is the logic, such as calculating the sizes of the optotypes and randomizing the order which they appear, and the view is the GUI (Graphical User Interface). The controller part is the part on the Android application that listens for button clicks and gestures. Usually the networking would fall under controller in the MVC pattern but was, in this system, chosen to be a separate larger structure. The reason the network part is separated from the controller is because it is large part of the system and it separates the project into three parts. The network part is also almost identical on both the serverside and the tabletside, so the code could easily be adapted for both. Figure \ref{system_structure} shows how the different parts are connected.

\begin{figure}[ht!]
\centering
\includegraphics[width=90mm]{images/system_structure.png}
\caption{Overview of the system structure\label{system_structure}}
\end{figure}

\subsection{Android application}
The Android application is written in Java, which is the official language for Android. The static parts of the GUI, which never changes unless application updates are released, are written in XML using Googles defined syntax \cite{android_gui}. This is similar to how HTML is written and is supported by Android natively.

The visual acuity charts on the Android device have its opotypes positions and sizes calculated at runtime before before they are displayed. If the charts would have been written in an XML file, it would have required that each optotype had its own XML element with a manually inserted size and position. This would result in very large and hard to read XML files and would require manual updating of each element if the sizes or positions should ever change. The charts are all drawn at the same time and on top of each other, but makes all but the chosen chart invisible. This makes it seem like only one chart is calculated at a time and avoids recalculating when switching back and fort between charts.

The optotypes provided to us by P.G. Söderberg were stored in the SVG (Scalable Vector Graphics) format which is a format that Android does not natively support. One possibility would be to convert the files to scalable fonts, but instead a third party library called AndroidSVG \cite{AndroidSVG} was used. This allowed us to avoid manipulating the optotype files and was considered to be the simplest solution.

% As Android does not natively have support for vector graphics and the chart optotype images are vector images the AndroidSVG library is used to display them \cite{AndroidSVG}.

The size of the optotypes is calculated using equation \ref{eq:optotype_size}, where $d$ is the distance in centimeters, $v$ is the angle in radians and $h$ is the height of the optotype in centimeters.

\begin{equation}
	\begin{split}
  		2d\ tan(\frac{v}{2}) & = h
  	\end{split}
  	\label{eq:optotype_size}
\end{equation}

The settings, such as the distance from the screen the patient is watching or the order which the optotypes appear, associated to each chart are also generated at runtime. If this were not the case, every different version of the chart would need its own XML file, as each version contains a different number of predefined charts.

In order to control the application, \textit{listeners} were created to listen to button clicks and gestures when the optician attempts to interact with the application. The listeners are Java classes that contains methods which are connected to an interactive object on the screen, such as a button. Whenever an optician interacts with an interactive object on the screen, the corresponding method in the listener is called. This method, in turn, calls other methods in order to perform the user requested action. The use of listeners separates the user input from the logic and creates a simple way of changing the functionality of an interactive object by letting the object call another listener method instead.

The design of the application follows the Android design principles by using Androids built in navigation drawer (fig. \ref{fig:example_nav_drawer}) and the built in views i.e. buttons, radio buttons, lists etc. \cite{android_design}.

\begin{figure}[htb!]
\centering
\begin{subfigure}{.45\textwidth}
  \centering
  %\includegraphics[width=.5\linewidth]{images/etdrs_top.png}
  \includegraphics[width=50mm]{images/no_nav_drawer.png}
  \caption{A closed navigation drawer which can be opened by clicking on the three lines on the top left of the screen or by swiping from the left edge of the screen to the right.}
\end{subfigure}\hfill
\begin{subfigure}{.45\textwidth}
  \centering
  \includegraphics[width=50mm]{images/nav_drawer.png}
  \caption{An open navigation drawer which can be closed by pressing the back button or by swiping from the right to the left.}
\end{subfigure}
\caption[Navigation drawer example]{An example on how Google Maps navigation drawer looks like. (Source: Christian Johansson)}
\label{fig:example_nav_drawer}
\end{figure}

\subsection{Server application}
The server application is written in Java and uses the built-in \textit{Swing} library for drawing the GUI and the images. Swing is a very flexible and easy-to-use library with a lot of great tools for creating a good looking GUI. In our case the server didn't need a very advanced GUI since the main purpose of the server is to render a fullscreen image at all times. This greatly simplifies the implementation of the GUI. Other than the rendering of images, the GUI consists of a small networking control class where the optician can specify a number of settings. These settings have to be configured to set what name the server has on the network. There are no other major settings at this time, but it was designed so that it can be easily expanded if future updates require more settings.

The rendering of images, specifically SVG vector graphic images, is handled by an \textit{~Apache} library called \textit{Batik}. It is a well supported and extensive library for rendering a multitude of different file formats. The SVG files contain information about the optotypes used on the charts. Using the same procedure as the tablet application to calculate the positions of the SVG images, we can easily create a well structured image to render to the screen.

When the networking part of the server receives information from the network it is submitted to a thread-safe queue. Separately from this, the GUI pulls the information from the queue and handles the information. This method is necessary because of how Swing is constructed. Swing runs on a dispatch-thread basis, meaning that there is a single event dispatching thread that handles all the incoming events and alters the GUI accordingly. Since the dispatch-thread decides when to update the screen attempting to alter the GUI from another thread simultaneously would result in unknown behaviour and possibly fatal errors, and is therefore prohibited. 

\subsection{Network}
The network communication between the client and the server consists of two parts. The first part is server detection and the second is server communication. All network communication are handled in threads separate from the main thread using Java's SwingWorker- and Androids AsyncTask class. Both classes allow us to perform network operations in the background, in a separate thread, and once the operations are complete the main thread of the program automatically gets the result.

\subsubsection{Server detection}
In order to detect all servers on the network, the Android application sends out a discovery request message, consisting of the string \\ \textit{AXANIVIS\_DISCOVER\_REQUEST}, 5 times with a 1 second interval using UDP (User Datagram Protocol) on the networks broadcast address using port 14141. The server, which constantly listen for broadcasts containing the correct message, then responds to that message with the message \textit{AXANIVIS\_DISCOVER\_RESPONSE\textbackslash nServerName}, again using UDP. Once the Android application receives a response it adds the server to a list. The optician can then choose to connect to one of the found servers or attempt the detection again if no servers were found.

Since UDP is a connectionless protocol, there is no guarantee that the server actually receives the message, the discovery request message is sent 5 times, which means the server have 5 chances to receive the message and respond to it. This means that the server receives a discovery request 0 to 5 times and respond to it the same number of times. To avoid duplicate server entries in the list, a server is only added if it doesn't already exist in the list.

\subsubsection{Server communication}
The communication between the Android application and the connected server is handled using strings of characters divided to three lines using line separators. The first line consists of a check word, which is the same for every message. If this check word is wrong or doesn't exist, the server ignores the message. The second line is the command to the server. This could be a command to set the current chart type or size to something else. The third line is the data required for the command. The data can be empty (null) or a string of characters. Depending on the command, the string of characters can be read as either a sequence of ASCII characters or a sequence of numbers. The messages defined for the server communications are commands to connect to the server, set the distance from the screen and set which chart to be displayed with which optotype. If a message with an incorrect command is sent to the server, the server ignores that message. If the wrong data is sent together with a correct command, the server keeps functioning but its output is undefined.

The messages are sent using a TCP (Transmission Control Protocol) connection to the server on port 14142. This guarantees that the entire message is correct and that it is retrieved by the server as long as the server is connected to the network. If a connection is timed out, the sent command is ignored and the optician has to reconnect to a server and resend the command.

%The communication between the client and server is handled using simple TCP (Transimission Control Protocol) messages. The TCP messages are used for sending and receiving commands and are structured as three lines of text (Example \ref{tcp_message}). The first line is a \textit{check} word and is the same for every \textit{command}. The second line is a \textit{command} word which tells the receiver what to do and the third line is the \textit{data} required for the \textit{command}. In many cases the simple \textit{command} is enough and the \textit{data} is left empty.

%\begin{center}
%\renewcommand{\lstlistingname}{Example}
%  \lstset{%
%    title=Example of TCP message,
%    basicstyle=\ttfamily\footnotesize\bfseries,
%    xleftmargin=.2\textwidth, xrightmargin=.2\textwidth
%  }
%\begin{lstlisting}[caption=TCP Message, label=tcp_message]
%CHECK_WORD
%LARGE_CHART
%NULL
%\end{lstlisting}
%\end{center}

\begin{figure}[h]
  \lstset{%
    basicstyle=\ttfamily\bfseries,
    xleftmargin=.4\textwidth, xrightmargin=.2\textwidth
  }
\begin{lstlisting}
AXANIVIS
DISTANCE
500
\end{lstlisting}
\caption{An example of how a network message that sets the distance from the screen to 500cm.
%A message with the check word \textit{AXANIVIS}, the command \mbox{\textit{DISTANCE}} and the data \textit{500}. This message would make the server to resize the currently active chart to be suitable for a patient watching from 500 centimeters. 
\label{fig:example_message}}
\end{figure}

%\section{Boundaries (OLD!)} 
%%%%%%%%%%%%%%%%%%%%%%%%%Avgränsningar: vad ska INTE göras, även om man kanske kunde tro det?
%We are not tasked to develop a library to display .SVG files, and will be using a licensed one. We are also not tasked to populate the database with symbols and layouts, but instead provide a comprehensive API so that clients can create their own.
%
%%%%%%%%%%%%%%vad ska bara göras om tid/resurser/omständigheter räcker till?
%
%There are however several extensions to the system that can be considered if we have time to spare. We can extend the system to support more types of charts, such as charts for testing colour blindness and stereoscopic vision, which also rely heavily on rooms with appropriate lighting.
%
%While it deviates further from the main part of the project, we were asked by our employer to consider implementing the ability to control devices such as lights around the room from inside the app. This could probably be done, and we will look into it if time allows.
%
%\section{Demands (OLD!)} %%%%%%%%%%%%%%%%%%%%%%%%Krav: vilka krav ställer ni/andra på ert resultat? hur snabbt? hur många användare? hur strömsnålt? eller vad som är relevant
%There are a few core demands that our product needs to meet. First and foremost, it needs to work, and do what the current testing system can already do. Second, it needs to be easy to use and have an intuitive interface. Third, it should be very energy efficient, since forcing the user to recharge the tablet often reduces productivity. 
%
%\section{Evaluation (OLD!)} %%%%%%%%%%%%%%%%%%%%%%%%Utvärdering: hur ska ni utvärdera ert arbete/system, hur vet ni om/hur bra ni lyckats?
%The easiest way to see if our product satisfies these demands is putting it into the hands of the opticians at the university hospital and letting them test it, recording bug reports or comments. Testing will also let us determine if the system is as energy efficient as we intended.

\chapter{ Results}
\section{System functionality} 
%The system fulfills all specified requirements and allow the user to quickly set up a chart on the monitor. No system crashes have occurred during testing, and no bugs that causes the system to not work properly have been encountered. With a stable WiFi connection, the system runs smoothly without any lag and the server detection finds a server in under a second. 
The resulting system ended up in three parts, an Android part, a server part and a network part, as shown in the system structure (fig. \ref{system_structure}). Every action on the Android device related to a chart is connected to the network part. Once an action has been chosen, the network part is called and it creates a message with the correct command and data and then sends this message to the server. Once the server receives this message, the server performs the requested actions.

\subsection{Android application}
The Android application uses Android's standard GUI elements. To navigate between the connection settings and different type of charts, Android's built in navigation drawer is used (fig. \ref{fig:app_nav_drawer}). In order to reach the connection settings, the optician must click the name of the connected server and in the case that no server is connected, it says, instead of the name of a server, \textit{Click to connect}.

\begin{figure}[ht!]
\centering
\includegraphics[width=56mm]{images/appgui/nav_drawer.png}
\caption[Android Navigation Drawer]{The Android application navigation drawer used for navigating the application. a) Name of the application b) The currently connected server c) Available optotypes d) Grating, not yet implemented e) Test, used for testing new functions}
\label{fig:app_nav_drawer}
\end{figure}

Once the connection settings page have been reached, the optician is shown the connection status and a \textit{scan network} button (fig. \ref{fig:scan1}) which can be used to scan the network for servers. Once the button is pressed, the button is disabled and found servers appears in a list below (fig. \ref{fig:scan2}). When the desired server has been found, the optician can click on it to connect to it and once the entire scan is complete, the \textit{scan network} button is enabled again.

\begin{figure}[ht!]
\centering
\begin{subfigure}{.5\textwidth}
  \centering
  %\includegraphics[width=.5\linewidth]{images/etdrs_top.png}
  \includegraphics[width=50mm]{images/appgui/scan1.png}
  \caption{No connection}
  \label{fig:scan1}
\end{subfigure}%
\begin{subfigure}{.5\textwidth}
  \centering
  \includegraphics[width=50mm]{images/appgui/scan2.png}
  \caption{Scan button pressed}
  \label{fig:scan2}
\end{subfigure}
\begin{subfigure}{.5\textwidth}
  \centering
  \includegraphics[width=50mm]{images/appgui/scan3.png}
  \caption{Connected to server}
  \label{fig:scan3}
\end{subfigure}
\caption{Server connection steps}
\label{fig:scan}
\end{figure}

Once the application has connected to a server, the optician can navigate to the chart that should be viewed on the screen. This is done by opening the navigation drawer, either by swiping from the left edge of the screen to the right, or by pressing the navigation drawer icon in the top left corner. When the navigation drawer is open, the optician can choose between the \textit{ETDRS} (Sloan letters), \textit{Tumbling E}, \textit{HVOT} or \textit{Icons} (LEA) to display that chart (fig. \ref{fig:etdrs_large}). 

\begin{figure}[ht!]
\centering
\includegraphics[width=120mm]{images/appgui/etdrs_large.png}
\caption{The upper large ETDRS 1 chart displayed with each rows LogMAR values to the far left.}
\label{fig:etdrs_large}
\end{figure}

The optician can now choose which chart version, size and distance to use. The chart version decides which optotype to use at which position. The ETDRS chart has 3 pre-made versions\cite{Ferris} and the rest of the charts have 1. There is also an option to use a randomly generated version which is generated when the application starts. If the random version is selected, the generate button is enabled. The random version is generated with no restrictions and should be used with caution.

The optician can also choose which chart size, \textit{Large}, \textit{Medium} or \textit{Small}, to use. The large chart is the same chart as the \textit{upper} and \textit{lower ETDRS} charts (fig. \ref{fig:etdrs_large}), the medium chart is the same as the \textit{one line chart} (fig. \ref{fig:icons_medium}) and the small chart is the same as the \textit{single optotype chart} (fig. \ref{fig:tumbling_e_small}). The reason the names in application differs from the names in the report is the result of an early design choice which most likely changes in the future. To navigate between the upper and lower chart, the optician must swipe the screen up or down and to page between the optotypes on the small chart, the optician must swipe left or right.

\begin{figure}[ht!]
\centering
\begin{subfigure}{.5\textwidth}
  \centering
  %\includegraphics[width=.5\linewidth]{images/etdrs_top.png}
  \includegraphics[width=60mm]{images/appgui/icons_medium.png}
  \caption{The medium Icons 1 chart}
  \label{fig:icons_medium}
\end{subfigure}%
\begin{subfigure}{.5\textwidth}
  \centering
  \includegraphics[width=60mm]{images/appgui/tumbling_e_small.png}
  \caption{The small Tumbling E 1 chart}
  \label{fig:tumbling_e_small}
\end{subfigure}
\caption{A medium and small chart}
\label{fig:chart_medium_small}
\end{figure}

The last setting the optician can choose is the distance setting. Here the optician can enter the distance a patient is from the screen, in centimeters, and then press \textit{Apply Distance}. This has no visible effect on the application but is used to set the optotype sizes on the monitor.

\subsection{Server}
The server application is in many ways very similar to the Android application. Both use the same methods for the calculations, such as the optotype sizes. The charts are built using the same code as the Android application with minor changes due to the fact that Java's Swing components have to be drawn rather than Android components. The monitor connected to the server only displays the currently chosen chart (fig. \ref{fig:server_charts}) and nothing else, such as LogMAR values or settings shown on the Android device.

\begin{figure}[ht!]
\centering
\begin{subfigure}{.5\textwidth}
  \centering
  %\includegraphics[width=.5\linewidth]{images/etdrs_top.png}
  \includegraphics[width=60mm]{images/servergui/etdrs_chart.png}
  \caption{The large ETDRS chart}
  \label{fig:server_large}
\end{subfigure}%
\begin{subfigure}{.5\textwidth}
  \centering
  \includegraphics[width=60mm]{images/servergui/etdrs_single_row.png}
  \caption{The medium ETDRS chart}
  \label{fig:server_medium}
\end{subfigure}
\caption{Example of charts as shown on the monitor connected to the server}
\label{fig:server_charts}
\end{figure}

\section{Tests}
The four tests presented in section \ref{sec:evaluation} were performed in order to find major faults with the system. All tests are considered to have met the requirements, though the usability could still be improved by making it more clear on how and where swipes can be performed.

\subsection{Developer walkthrough}
During the developer walkthrough (see section \ref{sec:evaluation}) of the system, all charts versions were found and all settings were usable on every chart by using table \ref{tab:test1_empty}. The connection settings were also reachable from everywhere in the application and both connecting and reconnecting was working properly.

\begin{table}[ht!]
\centering
\begin{tabular}{l c c c r}
Chart version		&	Large		&	Medium		&	Small		&	Distance	\\
\hline\hline
ETDRS 1				&	\checkmark	&	\checkmark	&	\checkmark	&	\checkmark	\\
ETDRS 2				&	\checkmark	&	\checkmark	&	\checkmark	&	\checkmark	\\
ETDRS R				&	\checkmark	&	\checkmark	&	\checkmark	&	\checkmark	\\
ETDRS Random		&	\checkmark	&	\checkmark	&	\checkmark	&	\checkmark	\\
\hline
Tumbling E 1		&	\checkmark	&	\checkmark	&	\checkmark	&	\checkmark	\\
Tumbling E Random	&	\checkmark	&	\checkmark	&	\checkmark	&	\checkmark	\\
\hline
HVOT 1				&	\checkmark	&	\checkmark	&	\checkmark	&	\checkmark	\\
HVOT Random			&	\checkmark	&	\checkmark	&	\checkmark	&	\checkmark	\\
\hline
Icons 1				&	\checkmark	&	\checkmark	&	\checkmark	&	\checkmark	\\
Icons Random		&	\checkmark	&	\checkmark	&	\checkmark	&	\checkmark	\\
\hline\hline
					&	Scanning	&	Connecting	& 	Reconnecting	& \\
Connection			&	\checkmark	&	\checkmark	&	\checkmark	&	\\
\hline\hline
\end{tabular}
\caption{Result of the developer walkthrough. A checkmark is placed in every area were a specific chart or connection setting has been reached.\label{tab:test1}}
\end{table}

\newpage
\subsection{Optotype Sizes}
The optotype sizes were calculated for and measured on a 15.6 inch computer screen and a 37 inch TV, both with 1920x1080 resolution. The size and resolution of the screens were hardcoded into the server before the tests were performed. As the TV is larger than the computer screen, and have the same resolution, each pixel is bigger and thus, the calculated size in pixels is smaller (table. \ref{tab:optotype_test}). The height in centimeters, rounded to two decimals, are both equal to the expected height and the test is considered successful.


\begin{table}[ht!]
\centering
\begin{tabular}{ p{1cm}|p{1cm}p{1cm}|p{1cm}p{1cm}|p{1cm} }
	 &	\multicolumn{2}{c}{\parbox{2cm}{\centering 15.6", 1920x1080}} & \multicolumn{2}{c}{\parbox{2cm}{\centering 37", 1920x1080}}	& 	Expected	\\
\hline
	 row  & px & cm & px & cm & cm \\
\hline\hline				
								1	&	162	&	2.91	&	68	&	2.91	&	2.91\\	
								2	&	128 & 	2.31	& 	54	&	2.31	&	2.31\\
		 						3	&	102 & 	1.84	& 	43	&	1.84	&	1.84\\
								4	&	81 	& 	1.46	& 	34	&	1.46	&	1.46\\
								5	&	64 	& 	1.16	& 	27	&	1.16	&	1.16\\
								6	&	51 	& 	0.92	& 	22	&	0.92	&	0.92\\
								7	&	41 	& 	0.73	& 	17	&	0.73	&	0.73\\
								8	&	32 	& 	0.58	& 	14	&	0.58	&	0.58\\
								9	&	26 	& 	0.46	& 	11	&	0.46	&	0.46\\
								10	&	20 	& 	0.37	& 	9	&	0.37	&	0.37\\
								11	&	16 	& 	0.29	& 	7	&	0.29	&	0.29\\
								12	&	13 	& 	0.23	& 	5	&	0.23	&	0.23\\
								13	&	10 	& 	0.18	& 	4	&	0.18	&	0.18\\
								14	&	8 	& 	0.15	& 	3	&	0.15	&	0.15\\
										\hline
\end{tabular}
\caption{Height of optotypes for a 15.6 and 37 inch screen, both with a 1920x1080 resolution with a distance setting of 200cm. The rightmost column displays the expected height of the optotypes in cm.} \label{tab:optotype_test}
\end{table}


\subsection{Usability}
The system usability was tested with the help of two persons with no previous background related to visual acuity charts or eye examinations. Since the system is only a prototype and the test is focused  usability, two test persons with no previous knowledge about visual acuity was enough to get an overview of the systems usability.

After a brief explanation to the different types of charts the test persons where asked to follow the procedures in \ref{app:usability}. None of the persons had no problem performing any of the tasks with the exception of one of them didn't realize that you had to swipe the screen in order to switch between the upper and lower charts in the application. This is not a major problem since it can easily be solved by an explanation or tutorial before using the application.



\subsection{Server Detection}
The server detection test was performed 3 times on a stable WiFi network. According to the results (Table. \ref{tab:server_detection_test}), every single one of the tests performed was successful in finding a server over 90\% of the time. The average success rate of the 3 tests was about 96.3\% which means that the server detection requirements are met.

\begin{table}[ht!]
\centering
\begin{tabular}{l c r}
Attempt	&	failures	&	succeses	\\
\hline
1	&	7	&	93	\\
2	&	1	&	99	\\
3	&	3	&	97	\\
\end{tabular}
\caption{The number of failures and successes during the server detection tests. \label{tab:server_detection_test}}
\end{table}

%\begin{figure}[ht!]
%\end{figure}

\section{Known bugs and issues}
Even though the system fulfils the requirements, it doesn't mean it is bug free. If the network is unstable or slow, the server can react several seconds later than the Android application. This can cause an optician to repeat an action which, in the end, results in the server acting on on both actions. In the case that the connection gets broken, all actions performed on the currently selected chart is ignored by the server, even after reconnecting. In this case, the optician can change to another chart size or version and then back again to resolve the issue.

\chapter{ Conclusion}

\section{Discussion}
Even though the system turned out as intended, there are things left to do before it can be used clinically. First of all, the system must be tested for more bugs and unintended behaviour. This should preferably be done by users other than the developers since the developers know how the system works and can't safely predict what another user attempts to do. The system must be tested by experts in the field of ophthalmology in order to make sure that the built system actually is a valid solution. This gives the system credibility and make it possible to use it clinically.

\section{Advantages of a digital system}
The advantages of a digital system is the ability to change the layout, optotypes and size of the chart without having to manually exchange charts. This saves both time and space as only one screen, a small server and a small Android device is necessary, while also providing easier access to specialized charts that helps patients with different needs. This system also allows the visual acuity tests to take place in rooms that previously were too small, as the system displays optotypes at a size suitable for the distance the patient is from the monitor.

\section{Disadvantages of a digital system}
A disadvantage with this system is that digital screens become very expensive at large sizes, a 40 inch LED TV, for example, can cost above 3500 SEK as of 30/8-2015 according to Elgiganten. The screen size limits the maximum distance a patient can watch the screen and the patient should preferably sit at least 4 meters away from screen \cite{PGSoderbergOral}. Another disadvantage is that the resolution, or rather pixel density, limits how close to a screen a patient can sit without the optotypes looking pixelated which makes it harder to read the optotype and result in an incorrect result from the test \cite{PGSoderbergOral}. Pixel density is a measurement of how many pixels (dots) can fit in a small section of the screen, usually denoted as Pixels Per Inch (PPI) or Dots Per Inch (DPI). The higher the DPI is the better the quality of small details and the optotype appears less pixelated.

\section{Alternative solutions}
A possible alternative solution to a custom server running on a Raspberry Pi or other computer would be to use a Chromecast instead. This could simplify the system for the optician as this would make the Android application work similar to the many media streaming applications that exists today. Using a Chromecast would also make sure the optician only have to worry about the Android application as the Chromecast is just a plug and play unit maintained by Google. A disadvantage to this is that the developer is required to have a Google developers account which costs a small sum of money (5\$ as of August 2015). This might not be a large sum of money but it was decided to be left for future exploration. Another disadvantage of using the Chromecast is that the application would need an internet connection instead of just a local network. Although screen mirroring is possible, the logical solution to create this system would be to use the Chromecasts \textit{Custom Media Receiver} which requires a webbapp located on the internet \cite{android_receiver_app}.

\section{Future Work}
Future work includes more charts and features for the system. One such feature is the grating acuity test, which allows testing of visual acuity on very young children or people with difficulties communicating. The test consists of several vertical lines with different widths and distances from each other. These lines are moving horizontally on a screen and if the patient can distinguish the lines, the patients eyes follows them. The eyesight can then be determined by looking on the thinnest line the patient can distinguish \cite{PGSoderbergOral}. Another feature is to include the option to, instead of changing the size of the optotypes, change the LogMAR values, so the top row doesn't necessarily start at 1.0 LogMAR. This way the server can calculate the largest size the top row optotypes can have and allow the chart to use the entire monitors screen no matter the distance from the patient to the screen.

Other charts to be included could be, for example, charts using the Landholt C, charts for testing color blindness or charts used for testing astigmatism. The random function for the charts should be improved with constraints, as it currently has none. These constraints would make sure that the same optotype, or to similar optotypes like \textit{C} and \textit{O}, are not repeated too many times in a row.


%\bibliography{references}{}
\renewcommand\thechapter{ }
\bibliography{IEEEabrv,references}
%\bibliographystyle{ieeetr}
\bibliographystyle{IEEEtran}

\newpage
\appendix
\chapter{ Testing} \label{App:test}
\section{Developer walkthrough}
\begin{table}[ht!]
\centering
\begin{tabular}{l c c c r}
Chart version		&	Large		&	Medium		&	Small		&	Distance	\\
\hline\hline
ETDRS 1				&		&		&		&		\\
ETDRS 2				&		&		&		&		\\
ETDRS R				&		&		&		&		\\
ETDRS Random		&		&		&		&		\\
\hline
Tumbling E 1		&		&		&		&		\\
Tumbling E Random	&		&		&		&		\\
\hline
HVOT 1				&		&		&		&		\\
HVOT Random			&		&		&		&		\\
\hline
Icons 1				&		&		&		&		\\
Icons Random		&		&		&		&		\\
\hline\hline
			&	Scanning	&	Connecting	& Reconnecting	& \\
Connection	&	&	&	& \\
\hline\hline
\end{tabular}
%\label{tab:test1_empty}
\caption{\label{tab:test1_empty}}
\end{table}

\newpage
\section{Usability test procedure \label{app:usability}}
\begin{enumerate}\bfseries

\item Start application

\item Connect to server

\item Display ETDRS 1 Large upper chart with a distance of 2 meters

\item Display ETDRS 1 Large lower chart with a distance of 2 meters

\item Display HVOT Random chart medium with a distance of 1.7 meters

\item Generate a new HVOT chart

\item Display Icons 1 small chart with a distance of 1.5 meters

\item Page through Icons 1 small chart with a distance of 1.5 meters

\item Reconnect to server

\end{enumerate}
\end{document}
